\documentclass[slidestop,mathserif,compress,xcolor=svgnames,table]{beamer} 
\mode<presentation>
{  
  \setbeamertemplate{background canvas}[vertical shading][bottom=blue!5,top=blue!5]
  \setbeamertemplate{navigation symbols}{}%{\insertsectionnavigationsymbol}
    \usetheme{BatonRouge}
}

\usepackage{pgf,pgfarrows,pgfnodes,pgfautomata,pgfheaps,pgfshade}
\usepackage{amsmath,amssymb,amsfonts}
\usepackage{multirow}
\usepackage{tabularx}
\usepackage{booktabs}
\usepackage{colortbl}
\usepackage{tikz}
\usetikzlibrary{shapes,arrows}
\usetikzlibrary{calc}
\pgfdeclarelayer{background}
\pgfdeclarelayer{foreground}
\pgfsetlayers{background,main,foreground}
\usepackage[latin1]{inputenc}
\usepackage{colortbl}
\usepackage[english]{babel}
\usepackage{hyperref}
\usepackage{movie15}
\hypersetup{
  pdftitle={HPC User Environment, Job Management with PBS/Loadleveler},
  pdfauthor={Alexander B. Pacheco, User Services Consultant, Louisiana State University}
}
%\usepackage{movie15}
\usepackage{times}

\setbeamercovered{dynamic}
\beamersetaveragebackground{DarkBlue!2}
\beamertemplateballitem

\usepackage[english]{babel}
\usepackage[latin1]{inputenc}
\usepackage[T1]{fontenc}
\usepackage{graphicx}
\definecolor{DarkGreen}{rgb}{0.0,0.3,0.0}
\definecolor{Blue}{rgb}{0.0,0.0,0.8} 
\definecolor{dodgerblue}{rgb}{0.1,0.1,1.0}
\definecolor{indigo}{rgb}{0.41,0.1,0.0}
\definecolor{seagreen}{rgb}{0.1,1.0,0.1}
\DeclareSymbolFont{extraup}{U}{zavm}{m}{n}
%\DeclareMathSymbol{\vardiamond}{\mathalpha}{extraup}{87}
\newcommand*\vardiamond{\textcolor{tigerspurple}{%
  \ensuremath{\blacklozenge}}}
\newcommand*\up{\textcolor{green}{%
  \ensuremath{\blacktriangle}}}
\newcommand*\down{\textcolor{red}{%
  \ensuremath{\blacktriangledown}}}
\newcommand*\const{\textcolor{darkgray}%
  {\textbf{--}}}


\setbeamercolor{uppercol}{fg=white,bg=red!30!black}%
\setbeamercolor{lowercol}{fg=black,bg=red!15!white}%
\setbeamercolor{uppercol1}{fg=white,bg=blue!30!black}%
\setbeamercolor{lowercol1}{fg=black,bg=blue!15!white}%%
\setbeamercolor{uppercol2}{fg=white,bg=green!30!black}%
\setbeamercolor{lowercol2}{fg=black,bg=green!15!white}%
\newenvironment{colorblock}[4]
{
\setbeamercolor{upperblock}{fg=#1,bg=#2}
\setbeamercolor{lowerblock}{fg=#3,bg=#4}
\begin{beamerboxesrounded}[upper=upperblock,lower=lowerblock,shadow=true]}
{\end{beamerboxesrounded}}
\newenvironment{ablock}[0]
{
\begin{beamerboxesrounded}[upper=uppercol,lower=lowercol,shadow=true]}
{\end{beamerboxesrounded}}
\newenvironment{bblock}[0]
{
\begin{beamerboxesrounded}[upper=uppercol1,lower=lowercol1,shadow=true]}
{\end{beamerboxesrounded}}
\newenvironment{eblock}[0]
{
\begin{beamerboxesrounded}[upper=uppercol2,lower=lowercol2,shadow=true]}
{\end{beamerboxesrounded}}


\title{HPC User Environment, Job Management with PBS/Loadleveler}


\author[Alex Pacheco]{\large{Alexander~B.~Pacheco}}
       
%\institute[High Performance Computing @ Louisiana State University - http://www.hpc.lsu.edu] {\inst{}\footnotesize{User Services Consultant\\LSU HPC \& LONI\\sys-help@loni.org}}
\institute[HPC Training: Fall 2011] {\inst{}\footnotesize{User Services Consultant\\LSU HPC \& LONI\\sys-help@loni.org}}

\date[\hfill{Sep 14, 2011\hspace{2cm}\insertframenumber/\inserttotalframenumber}]{\scriptsize{HPC Training Fall 2011\\Louisiana State University\\Baton Rouge\\Sep 14, 2011}}
     
\subject{Talks}
\keywords{LONI \& LSU HPC Computing Resources, User Environment, Job Management, PBS, Loadleveler}
% This is only inserted into the PDF information catalog. Can be left
% out. 




% If you have a file called "university-logo-filename.xxx", where xxx
% is a graphic format that can be processed by latex or pdflatex,
% resp., then you can add a logo as follows:

\pgfdeclareimage[height=1.0cm]{loni-logo}{LONI}
%\pgfdeclareimage[height=0.5cm]{gold-logo}{GOLD_LSU}
%\pgfdeclareimage[height=0.8cm]{purple-logo}{PURPLELSU}
\pgfdeclareimage[height=1.2cm]{purple-logo}{ITSbanner}
%\pgfdeclareimage[height=1.5cm]{lsuccttower-logo}{FULLCOLORTOWER_VERT}%{qtp}
%\pgfdeclareimage[height=0.5cm]{lsutower-logo}{ProcessHorizontal}
% \logo{\pgfuseimage{purple-logo}}

% put MSRI logo in bottom left
\setbeamertemplate{sidebar left}{%
   \vfill%
   \rlap{\hskip0.02cm%
         %
         {\pgfuseimage{loni-logo}}}%
   \vskip-7pt%
   \llap{\usebeamertemplate***{navigation symbols}\hskip0.1cm}%
   \vskip2pt%
}
\setbeamertemplate{sidebar right}{%
   \vfill%
   \rlap{\hskip-1.55cm%
         %
         {\pgfuseimage{purple-logo}}}%
   \vskip-6pt%
   \llap{\usebeamertemplate***{navigation symbols}\hskip0.1cm}%
   \vskip2pt%
}

% Delete this, if you do not want the table of contents to pop up at
% the beginning of each subsection:
 \AtBeginSection[]
 {
   \begin{frame}<beamer>
    \frametitle{\small{Outline}}
     \small
     \tableofcontents[currentsection,currentsubsection]
   \end{frame}
 }

\begin{document}

\frame{\titlepage}

\normalsize
\begin{frame}[label=toc,squeeze]
  \footnotesize
  \frametitle{\small{Outline}}
  \tableofcontents
\end{frame}

\normalsize

%\part{Introduction}
\section{Hardware Overview}

\begin{frame}
\frametitle{\small LONI \& LSU HPC Clusters}
\begin{itemize}
\item Two major architectures.
\end{itemize}
\begin{columns}
\column{0.5\textwidth}
\begin{colorblock}{white}{blue!30!black}{black}{blue!15!white}{\bf Linux Clusters}
\begin{itemize}
\item Vendor: Dell
\item OS: Red Hat 
\item CPU: Intel Xeon
\end{itemize}
\end{colorblock}
\column{0.5\textwidth}
\begin{colorblock}{white}{blue!30!black}{black}{blue!15!white}{\bf AIX Clusters}
\begin{itemize}
\item Vendor: IBM
\item OS: AIX 
\item CPU: Power 5/7
\end{itemize}
\end{colorblock}
\end{columns}
\vspace{1cm}
\begin{itemize}
\item The LONI AIX clusters are on a path to decommissioning.
\end{itemize}
\end{frame}

\begin{frame}
\scriptsize{
\begin{block}{Linux Clusters}
\begin{center}
\begin{tabular}{|c|c|c|c|c|c|}
\hline
 & Name & Peak TeraFLOPS/s & Location & Status & Login\\
\hline
 \multirow{6}{*}{LONI} & QueenBee & 50.7 & ISB & Production & LONI \\
                                 & Eric & 4.7 & LSU & Production & LONI\\
                                 & Louie & 4.7 & Tulane & Production & LONI\\
                                 & Oliver & 4.7 & ULL & Production & LONI\\
                                 & Painter & 4.7 & LaTech & Production & LONI\\
                                 & Poseidon & 4.7 & UNO & Production & LONI\\
\hline
\multirow{2}{*}{ LSU HPC} & Tezpur & 15.3 & LSU & Production & HPC\\
                                       & Philip & 3.5 & LSU & Production & HPC\\
\hline
\end{tabular}
\end{center}
\end{block}
\begin{block}{AIX Clusters}
\begin{center}
\def\firstrowcolor{\rowcolor{green}}
\def\secondrowcolor{\rowcolor{blue!50}}
\def\thirdrowcolor{\rowcolor{tigerspurple!80}}
\begin{tabular}{|c|c|c|c|c|c|}
\hline
 & Name & Peak TF/s & Location & Status & Login\\
\hline
 \multirow{5}{*}{LONI} & Bluedawg & 0.85 & LaTech & Production & LONI \\
                                  & Ducky & 0.85 & UNO & 9/30/2011 & LONI\\
                                  & Lacumba & 0.85 & Southern & 12/22/2011 & LONI\\
                                  & Neptune & 0.85 & Tulane & 9/30/2011 & LONI\\
                                  & Zeke & 0.85 & ULL & 9/30/2011 & LONI\\
\hline
\multirow{2}{*}{ LSU HPC} & Pelican & 2.6 & LSU & Production & HPC\\
                                       & Pandora & 6.8 & LSU & Production & HPC\\
\hline
\end{tabular}
\end{center}
\end{block}
}
\end{frame}


\begin{frame}
\frametitle{\small Account Management}
\begin{bblock}{}
\begin{itemize}
\item LONI account
\item[] \url{https://allocations.loni.org}
\item LSU HPC account
\item[] \url{https://accounts.hpc.lsu.edu}
\item All LONI AIX clusters are being decommissioned.
\item Newest cluster at LSU HPC is Pandora.
\end{itemize}
\end{bblock}

\begin{eblock}{}
 \begin{itemize}
  \item The default Login shell is bash
  \item Supported Shells: bash, tcsh, ksh, csh \& sh
  \item Change Login Shell at the profile page
 \end{itemize}
\end{eblock}

\end{frame}

\begin{frame}
\frametitle{\small Cluster Architecture}
\begin{columns}
\column{4cm}
\begin{itemize}
\item A cluster is a group of computers (nodes) that works together closely
\item Type of nodes
\begin{enumerate}
{\scriptsize
\item[$\vardiamond$] Head node
\item[$\vardiamond$] Compute node
}
\end{enumerate}
\end{itemize}
\column{7cm}
\begin{center}
\includegraphics[width=1.05\textwidth,clip=true]{cluster}
\end{center}
\end{columns}
\end{frame}

\begin{frame}
\frametitle{\small LINUX Cluster Hardware}
\begin{itemize}
\item Queen Bee 
\begin{enumerate}
{\scriptsize
\item[$\blacklozenge$]668 nodes: 8 Intel Xeon cores @ 2.33 GHz
\item[$\blackdiamond$]8 GB RAM
\item[$\vardiamond$]192 TB storage
}
\end{enumerate}
\item Other LONI Linux clusters
\begin{enumerate}
{\scriptsize
\item[$\vardiamond$]128 nodes: 4 Intel Xeons cores @ 2.33 GHz
\item[$\vardiamond$]4 GB RAM
\item[$\vardiamond$]9 TB storage
}
\end{enumerate}
\item Tezpur
\begin{enumerate}
{\scriptsize
\item[$\vardiamond$]360 nodes, 4 Intel Xeon cores @ 2.33 GHz
\item[$\vardiamond$]4 GB RAM
\item[$\vardiamond$]32 TB storage
}
\end{enumerate}
\item Philip
\begin{enumerate}
{\scriptsize
\item[$\vardiamond$]37 nodes, 8 Intel Xeon cores @ 2.93 GHz
\item[$\vardiamond$]24/48/96 GB RAM
\item[$\vardiamond$]Shares storage with Tezpur
}
\end{enumerate}
\end{itemize}
\end{frame}

\begin{frame}
\frametitle{\small AIX Cluster Hardware}
\begin{itemize}
\item LONI AIX clusters
\begin{enumerate}
{\scriptsize
\item[$\vardiamond$]14 Power5 nodes, 8 IBM Power5 processors @ 1.9 GHz per node
\item[$\vardiamond$]16 GB RAM
\item[$\vardiamond$]280 GB storage
}
\end{enumerate}
\item Pelican
\begin{enumerate}
{\scriptsize
\item[$\vardiamond$]16 Power5+ nodes, 16 IBM Power5+ processors @ 1.9 GHz per node
\item[$\vardiamond$]32 GB RAM
\item[$\vardiamond$]21 TB strorage
}
\end{enumerate}
\item Pandora
\begin{enumerate}
{\scriptsize
\item[$\vardiamond$]8 Power7 nodes, 8 IBM Power7 processors @ 7.33 GHz per node
\item[$\vardiamond$]128 GB RAM
\item[$\vardiamond$]19 TB storage
}
\end{enumerate}
\end{itemize}
\end{frame}

\begin{frame}
\frametitle{\small Why is Cluster Hardware important?} 
\begin{columns}
 \column{5cm}
\begin{itemize}
 \item There are numerous different architectures in the HPC world.
 \item Choose the software to install or use depending on cluster architecture.
 \begin{enumerate}
  \item Linux: EM64T, AMD64, X86\_64
  \item AIX: Power5, Power7
 \end{enumerate}
\end{itemize}
\column{7cm}
\vspace{-1cm}
\begin{center}
\includegraphics[width=\textwidth,clip=true]{namd-down}
\end{center}
\end{columns}
\end{frame}

\begin{frame}
\frametitle{\small Usage: Max Memory}
\begin{itemize}
\item The amount of installed memory less the amount that is used by the operating system and other utilities 
\item Max amount per node
\begin{enumerate}
{\scriptsize
\item[$\vardiamond$]Linux clusters: $\sim$6 GB for Queen Bee, $\sim$3 GB for others
\item[$\vardiamond$]AIX clusters: $\sim$13 GB  
\item[$\vardiamond$]Pandora:$\sim$125 GBstorage
}
\end{enumerate}
\end{itemize}
\end{frame}


\section{User Environment}
\subsection{Accessing LONI \& LSU HPC clusters}
\begin{frame}
\frametitle{\small Accessing LONI \& LSU HPC clusters}
\begin{itemize}
\item LONI Host name: \texttt{<cluster name>.loni.org}
\begin{itemize}
\item[$\bigstar$] Eric: eric.loni.org
\end{itemize}
\item LSU HPC Host name: \texttt{<cluster name>.hpc.lsu.edu}
\begin{itemize}
\item[$\bigstar$] Tezpur: tezpur.hpc.lsu.edu
\end{itemize}
\item Use ssh to connect
\begin{itemize}
\item[$\bigstar$] $\ast$nix and Mac: \texttt{ssh <host name>}
\item[$\bigstar$] Windows: use Putty, Secure Shell Client or Bitvise Tunnelier
\end{itemize}
\item The default Login shell is bash
\item Supported shells: bash, tcsh, ksh, csh \& sh
\item Change the login shell at the profile page
\begin{enumerate}
{\scriptsize
\item[$\vardiamond$] LONI: \url{https://allocations.loni.org}
\item[$\vardiamond$] LSU HPC: \url{https://accounts.hpc.lsu.edu}
}
\end{enumerate}
\item Reset your password
\begin{enumerate}
{\scriptsize
\item[$\vardiamond$] LONI: \url{https://allocations.loni.org/user\_reset.php}
\item[$\vardiamond$] LSU HPC: \url{https://accounts.hpc.lsu.edu/user\_reset.php} 
}
\end{enumerate}
\end{itemize}
\end{frame}

\begin{frame}
\frametitle{\small Connecting to Eric from a Linux box}
\vspace{-1.5cm}
\begin{columns}
\column{12cm}
\begin{center}
\includegraphics[width=1.35\textheight,clip=true]{X11-Linux}
\end{center}
\end{columns}
\end{frame}

\begin{frame}
\frametitle{\small Connecting to Eric from a Mac box}
\vspace{-1.5cm}
\begin{columns}
\column{12cm}
\begin{center}
\includegraphics[width=1.35\textheight,clip=true]{X11-mac}
\end{center}
\end{columns}
\end{frame}

\begin{frame}
\frametitle{\small Connecting to Eric from a Windows box}
\begin{itemize}
\item Download and Install
\begin{enumerate}
{\scriptsize
\item X-Server: X-ming \url{http://www.straightrunning.com/XmingNotes/}
\item SSH Client: Putty \url{http://www.chiark.greenend.org.uk/~sgtatham/putty/}
\item SSH+SFTP/SCP Client: Bitvise Tunnelier \url{http://www.bitvise.com/tunnelier}
}
\end{enumerate}
\end{itemize}
\end{frame}

\begin{frame}
\frametitle{\small Start X-ming}
\begin{center}
\includegraphics[width=0.2\textwidth,clip=true]{X-ming-desk}
\includegraphics[width=0.3\textwidth,clip=true]{X-ming-tray}
\end{center}
\end{frame}

\begin{frame}
\frametitle{\small Putty with X11}
\begin{center}
\only<1>{\includegraphics[width=0.5\textwidth,clip=true]{Putty-X11}}
\only<2>{\includegraphics[width=0.5\textwidth,clip=true]{Putty-Login}}
\only<3>{\includegraphics[width=0.75\textwidth,clip=true]{Putty-SSH-Key}}
\only<4>{\includegraphics[width=0.75\textwidth,clip=true]{Putty-UID}}
\only<5>{\includegraphics[width=0.75\textwidth,clip=true]{Putty-Loggedin}}
\only<6>{\includegraphics[width=0.75\textheight,clip=true]{Putty-Jmol}}
\end{center}
\end{frame}

\begin{frame}
\frametitle{\small Configure Tunnelier/SSH Client to Tunnel X11 Connections}
\begin{center}
\only<1>{\includegraphics[width=0.75\textwidth,clip=true]{SSH-Secure-Shell}}
\only<2>{\includegraphics[width=0.75\textheight,clip=true]{Tunnelier-X11}}
\end{center}
\end{frame}

\subsection{File Systems}
\begin{frame}
\frametitle{\small File Systems}
\scriptsize{
\begin{columns}
\column{11cm}
\begin{block}{}
\begin{center}
\begin{tabular}{|c|p{1.5cm}|c|c|p{2.5cm}|}
\hline
& Distributed File System & Throughput & File life time & Best used for \\
\hline
{Home} & Yes & Low & Unlimited & {Code in development, compiled executable}\\ 
\hline
Work & Yes & High & 30 days & Job input/output \\
\hline
{Local Scratch} & No & & Job Duration & Temporary files \\
\hline
\end{tabular}
\end{center}
\end{block}
\end{columns}
}
\begin{itemize}
\item {\footnotesize Tips}
\begin{enumerate}
\item[$\vardiamond$] Never write job output to your home directory
\item[$\vardiamond$] Do not write temporary files to /tmp, use local scratch or work space
\item[$\vardiamond$] Work space is not for long term storage. Files are purged peridocally
\item[$\vardiamond$] Use \texttt{rmpurge} to delete large amount of files.
\end{enumerate}
\end{itemize}
\end{frame}

\begin{frame}
\frametitle{\small Disk Quota}
{\scriptsize
\begin{columns}
\column{12cm}
\begin{block}{}
\begin{center}
\begin{tabular}{|c|c|c|c|c|c|}
\hline
\multirow{2}{*}{Cluster} & \multicolumn{2}{c|}{Home} & \multicolumn{2}{c|}{Work} & Scratch \\
\cline{2-6}
 & Access Point & Quota & Access Point & Quota & Access Point \\
\hline
LONI Linux & /home/\$USER & 5GB & /scratch/\$USER & 100GB & /var/scratch\\
\hline
LONI AIX & /home/\$USER & 500MB & /work/default/\$USER & 20GB & /var/scratch\\
\hline
HPC Linux & /home/\$USER & 5GB & /work/\$USER & NA & /var/scratch\\
\hline
HPC AIX & /home/\$USER & 5GB & /work/\$USER & 50GB & /scratch/local\\
\hline
\end{tabular}
\end{center}
\end{block}
\end{columns}
}
{\footnotesize
\begin{itemize}
\item No quota is enforced on the work space of QueenBee
\item Work directory is created within an hour of first login
\item Check current disk usage
\item[] Linux: \texttt{showquota}
\item[] AIX: \texttt{quota}
\end{itemize}
}
\end{frame}

\begin{frame}
\begin{eblock}{Exercise 1}
 \begin{itemize}
  \item Log in to any cluster
  \item Check your disk quota
  \begin{enumerate}
    \item Linux: \texttt{showquota}
    \item AIX: \texttt{quota}
  \end{enumerate}
  \item Copy the traininglab directory
  \item[] \texttt{cp -r /home/apacheco/traininglab .}
 \end{itemize}
\end{eblock}

\begin{ablock}{}
 \begin{itemize}
  \item If you are not familiar with working on a Linux/Unix system
  \begin{enumerate}
   \item Loni Moodle course @ \url{https://docs.loni.org/moodle}: HPC104 or HPC105
  \end{enumerate}

 \end{itemize}

\end{ablock}


\end{frame}


\subsection{Software Management}
\begin{frame}
\frametitle{\small Managing User Environment}
\begin{itemize}
\item Environment variables
\begin{enumerate}
\item[$\vardiamond$]PATH: where to look for executables
\item[$\vardiamond$]LD\_LIBRARY\_PATH: where to look for shared libraries
\item[$\vardiamond$]Other custom environment variables needed by various software
\end{enumerate}
\item {\bf SOFTENV} is a software that is used to set up these environment variables on all the clusters
\begin{enumerate}
\item[$\vardiamond$]More convenient than setting numerous environment variables in .bashrc or .cshrc
\end{enumerate}
\end{itemize}
\end{frame}

\begin{frame}[fragile]
\frametitle{\small Listing All packages}
\begin{itemize}
\item Command \texttt{softenv} lists all packages that are managed by {\bf SOFTENV}
\end{itemize}
{\tiny
\begin{alertblock}{}
\begin{verbatim}
[apacheco@eric2 ~]$ softenv
SoftEnv version 1.6.2
...
----------------------------------------------------------------------------

These are the macros available:

*   @default                      


These are the keywords explicitly available:

    +ImageMagick-6.4.6.9-intel-11.1
                                   @types: Applications Visualization @name:
                                    ...
    +NAMD-2.6-intel-11.1-mvapich-1.1
                                   @types: Applications @name: NAMD @version:
                                     ...
    +NAMD-2.7b2-intel-11.1-mvapich-1.1
                                   @types: Applications @name: NAMD @version:
                                     ...
\end{verbatim}
\end{alertblock}
}
\end{frame}

\begin{frame}[fragile]
\frametitle{\small Searching for a Specific Package}
\begin{itemize}
\item Use \texttt{-k} option with \texttt{softenv}
\end{itemize}
{\tiny
\begin{alertblock}{}
\begin{verbatim}
[apacheco@eric2 ~]$ softenv -k gaussian
SoftEnv version 1.6.2
...
Search Regexp: gaussian
----------------------------------------------------------------------------

These are the macros available:

These are the keywords explicitly available:

    +gaussian-03                   @types: Applications Chemistry @name:
                                     Gaussian @version: 03 @build: @internal:
                                     ...
    +gaussian-09                   @types: Applications Chemistry @name:
                                     Gaussian @version: 09 @build: @internal:
                                     ...
    +gaussview-4.1.2               @types: Applications Chemistry @name:
                                     GaussView @version: 4.1.2 @build: - @about:
                                     ...

These are the keywords that are part of the software tree,
however, it is not suggested that you use these:
\end{verbatim}
\end{alertblock}
}
\end{frame}

\begin{frame}[fragile]
\frametitle{\small Setting up Environment via Softenv: One Time Change}
\begin{itemize}
\item Setting up environment variables to use a certain package in the current session only.
\begin{enumerate}
{\scriptsize
\item[$\vardiamond$] Remove a package: \texttt{soft add <key>}
\item[$\vardiamond$] Add a package: \texttt{soft add <key>}
}
\end{enumerate}
\end{itemize}
{\tiny
\begin{alertblock}{}
\begin{verbatim}
[apacheco@eric2 ~]$ which g09
/usr/local/packages/gaussian09/g09/g09
[apacheco@eric2 ~]$ soft delete +gaussian-09
[apacheco@eric2 ~]$ which g09
/usr/bin/which: no g09 in (/home/apacheco/bin:...
[apacheco@eric2 ~]$ soft add +gaussian-03
[apacheco@eric2 ~]$ which g03
/usr/local/packages/gaussian03/g03/g03
\end{verbatim}
\end{alertblock}
}
\end{frame}

\begin{frame}[fragile]
\frametitle{\small Setting up Environment via Softenv: Permanent Change}
\begin{itemize}
\item Setting up the environment variables to use a certain software package(s).
\begin{enumerate}
{\footnotesize
\item[$\vardiamond$] First add the key to \texttt{\$HOME/.soft}.
\item[$\vardiamond$] Execute \texttt{resoft} at the command line.
}
\end{enumerate}
\end{itemize}
{\tiny
\begin{alertblock}{}
\begin{verbatim}
[apacheco@eric2 ~]$ cat .soft
#   
# This is the .soft file.
...
+mvapich-1.1-intel-11.1
+intel-fc-11.1
+intel-cc-11.1
+espresso-4.3.1-intel-11.1-mvapich-1.1
+gaussian-09
+lmto-intel-11.1
+nciplot-intel-11.1
+gaussview-4.1.2
+jmol-12
+vmd-1.8.6
+xcrysden-1.5.24-gcc-4.3.2
+tcl-8.5.8-intel-11.1
+gamess-12Jan2009R1-intel-11.1
+nwchem-5.1.1-intel-11.1-mvapich-1.1
+cpmd-3.11.1-intel-11.1-mvapich-1.1
@default
[apacheco@eric2 ~]$ resoft
\end{verbatim}
\end{alertblock}
}
\end{frame}

\begin{frame}[fragile]
\frametitle{\small Querying a Softenv Key}
\begin{itemize}
\item \texttt{soft-dbq} shows which variables are set by a SOFTENV key
\end{itemize}
{\tiny
\begin{alertblock}{}
\begin{verbatim}
[apacheco@eric2 ~]$ soft-dbq +espresso-4.3.1-intel-11.1-mvapich-1.1

This is all the information associated with
the key or macro +espresso-4.3.1-intel-11.1-mvapich-1.1.

-------------------------------------------

Name: +espresso-4.3.1-intel-11.1-mvapich-1.1
Description: @types: Applications @name: Quantum Espresso @version: 4.3.1 @build: ...
Flags: none
Groups: none
Exists on: Linux 

-------------------------------------------

On the Linux architecture,
the following will be done to the environment:

  The following environment changes will be made:
    ESPRESSO_PSEUDO = /usr/local/packages/espresso/4.3.1/intel-11.1-mvapich-1.1/pseudo
    ESPRESSO_ROOT = /usr/local/packages/espresso/4.3.1/intel-11.1-mvapich-1.1
    ESPRESSO_TMPDIR = /work/${USER}
    PATH = ${PATH}:/usr/local/packages/espresso/4.3.1/intel-11.1-mvapich-1.1/bin

-------------------------------------------

\end{verbatim}
\end{alertblock}
}
\end{frame}

\begin{frame}<1,2>
\begin{eblock}{Exercise 2: Use Softenv}
 \begin{itemize}
  \only<1->{\item Find the key for VISIT (a visualization package).}
  \visible<2>{\item[] \texttt{softenv -k visit}}
  \only<1->{\item Check what variables are set through the key.}
  \visible<2>{\item[] \texttt{soft-dbq +visit}}
  \only<1->{\item Set up your environment to use VISIT.}
  \visible<2>{\item[] \texttt{soft add +visit}}
  \only<1->{\item Check if the variables are correctly set by using \texttt{which visit}.}
  \visible<2>{\item[] \texttt{/usr/local/packages/visit/bin/visit}}
 \end{itemize}

\end{eblock}

\end{frame}


\begin{frame}
\frametitle{\small Compilers}
\begin{colorblock}{white}{blue!30!black}{black}{blue!15!white}{}
\begin{center}
\begin{tabular}{|c|c|c|c|c|}
\hline
\multirow{2}{*}{Language} & \multicolumn{3}{c|}{Linux Cluster} & AIX Clusters\\
\cline{2-5}
 & Intel & PGI & GNU & XL \\
\hline
Fortran & ifort & pgf77,pgf90 & gfortran & xlf,xlf90\\\hline
C & icc & pgcc & gcc & xlc\\\hline
C++ & icpc & pgCC & g++ & xlC\\
\hline
\end{tabular}
\end{center}
\end{colorblock}
\begin{itemize}
\item Usage: <compiler> <options> <your\_code>
\begin{enumerate}
\item[$\vardiamond$] Example: icc -O3 -o myexec mycode.c
\end{enumerate}
\item Some compilers options are architecture specific
\begin{enumerate}
\item[$\vardiamond$] Linux: EM64T, AMD64 or X86\_64
\item[$\vardiamond$] AIX: power5,power7 or powerpc
\end{enumerate}
\end{itemize}
\end{frame}

\begin{frame}[allowframebreaks,fragile]
\frametitle{\small Compilers for MPI programs}
\begin{colorblock}{white}{blue!30!black}{black}{blue!15!white}{}
\begin{center}
\begin{tabular}{|c|c|c|}
\hline
Language & {Linux Cluster} & AIX Clusters\\
\hline
Fortran & mpif77,mpif90 & mpxlf,mpxlf90\\\hline
C & mpicc & mpcc\\\hline
C++ & mpiCC & mpCC\\
\hline
\end{tabular}
\end{center}
\end{colorblock}
\begin{itemize}
\item Usage: <compiler> <options> <your\_code>
\begin{enumerate}
\item[$\vardiamond$] Example: mpif90 -O2 -o myexec mycode.f90
\end{enumerate}
\item On Linux clusters
\begin{enumerate}
\item[$\vardiamond$] Only one compiler for each language
\item[$\vardiamond$] There is no intel\_mpicc or pg\_mpicc
\end{enumerate}
\item There are many different versions of MPI compilers on Linux clusters
\begin{enumerate}
\item[$\vardiamond$]  Each of them is built around a specific compiler
\item[$\vardiamond$]  Intel, PGI or GNU
\end{enumerate}
\item It is extremely important to compile and run you code with the same version!!!
\item Use the default version if possible
\item These MPI compilers are actually wrappers
\begin{enumerate}
\item[$\vardiamond$]  They still use the compilers we've seen on the previous slide
\begin{enumerate}
\item[$\bigstar$]  Intel, PGI or GNU
\end{enumerate}
\item[$\vardiamond$]  They take care of everything we need to build MPI codes
\begin{enumerate}
\item[$\bigstar$]  Head files, libraries etc.
\end{enumerate}
\item[$\vardiamond$]  What they actually do can be reveal by the \texttt{-show} option
\end{enumerate}
\end{itemize}
{\tiny
\begin{alertblock}{}
{%\color{black}
\begin{verbatim}
[apacheco@eric2 ~]$ mpif90 -show
ln -s /usr/local/packages/mvapich/1.1/intel-11.1/include/mpif.h mpif.h
ifort -fPIC -L/usr/local/ofed/lib64 -Wl,-rpath-link -Wl, \
   /usr/local/packages/mvapich/1.1/intel-11.1/lib/shared \
  -L/usr/local/packages/mvapich/1.1/intel-11.1/lib/shared \
  -L/usr/local/packages/mvapich/1.1/intel-11.1/lib \
  -lmpichf90nc -lmpichfarg -lmpich -L/usr/local/ofed/lib64 \
  -Wl,-rpath=/usr/local/ofed/lib64 -libverbs -libumad -lpthread -lpthread -lrt -limf
rm -f mpif.h
\end{verbatim}
}
\end{alertblock}
}
\end{frame}

\begin{frame}
\frametitle{\small Application Packages}
\begin{itemize}
\item Installed under \texttt{/usr/local/packages}
\item Most of them managed by SOFTENV
\begin{itemize}
\item[$\vardiamond$] Numerical and utility libraries
\begin{enumerate}
\item[$\blacksquare$] FFTW, HDF5, NetCDF, PetSc, Intel MKL
\end{enumerate}
\item[$\vardiamond$] Computational Chemistry
\begin{enumerate}
\item[$\blacksquare$] Amber, CPMD, Gaussian, GAMESS, Gromacs, LAMMPS, NAMD, NWCHEM
\end{enumerate}
\item[$\vardiamond$] Visualization
\begin{enumerate}
\item[$\blacksquare$] GaussView, VisIt, VMD 
\end{enumerate}
\item[$\vardiamond$] Profiling/debugging tools
\begin{enumerate}
\item[$\blacksquare$] DDT, Tau, TotalView
\end{enumerate}
\item[$\vardiamond$] $\cdots$
\end{itemize}
\end{itemize}
\end{frame}

\begin{frame}
 \begin{eblock}{Exercise 3: Compiling a code}
  \begin{enumerate}
  \item Serial Code
  \begin{itemize}
   \only<1->{\item On Linux cluster, add the soft keys for either Intel (\texttt{+intel-fc-11.1}) or GCC (\texttt{+gcc-4.3.2})}
   \only<1->{\item Compile \texttt{hello.f90} with a compiler of your choice}
   \visible<2>{\item[] \texttt{ifort -o hello hello.f90}}
   \visible<2>{\item[] \texttt{xlf90 -o hello hello.f90}}
   \only<1->{\item Run the executable from the command line}
   \visible<2>{\item[] \texttt{./hello}}
  \end{itemize}
  \item Parallel Code
  \begin{itemize}
   \only<1->{\item On Linux cluster, find the appropriate key for mpi implementation of the above compiler}
   \only<1->{\item Compile \texttt{hello\_mpi.f90}}
   \visible<2>{\item[] \texttt{mpif90 -o hellompi hello\_mpi.f90}}
   \only<1->{\item Do Not run the parallel code, we'll use a script to submit to a job manager}
  \end{itemize}

  \end{enumerate}
 \end{eblock}

\end{frame}

\section{Job Management}
\subsection{Queues}
\begin{frame}
\frametitle{\small The Cluster Environment}
\begin{columns}
\column{4cm}
\begin{itemize}
\item A cluster is a group of computers (nodes) that works together closely
\item Type of nodes
\begin{enumerate}
{\scriptsize
\item[$\vardiamond$] Head node
\item[$\vardiamond$] Multiple Compute nodes
}
\end{enumerate}
\item Multi User Environment
\item Each user may have multiple jobs running simultaneously.
\end{itemize}
\column{7cm}
\begin{center}
\includegraphics[width=1.05\textwidth,clip=true]{cluster}
\end{center}
\end{columns}
\end{frame}

\begin{frame}
\frametitle{\small Batch Queuing System}
\begin{colorblock}{white}{blue!30!black}{black}{blue!15!white}{}
\begin{itemize}
\item A software that manages resources (CPU time, memory, etc) and schedules job execution
\begin{enumerate}
\item[$\vardiamond$] Linux Clusters: Portable Batch System (PBS)
\item[$\vardiamond$] AIX Clusters: Loadleveler
\end{enumerate}
\item A job can be considered as a user's request to use a certain amount of resources for a certain amount of time
\item The batch queuing system determines
\begin{enumerate}
\item The order jobs are executed
\item On which node(s) jobs are executed
\end{enumerate}
\end{itemize}
\end{colorblock}
\end{frame}

\begin{frame}
\frametitle{\small A Simplified View of Job Scheduling}
\begin{columns}
\column{7cm}
\vspace{-1cm}
\begin{center}
\includegraphics[width=0.9\textwidth,clip=true]{JobSchedule-1}
\end{center}
\column{5cm}
\vspace{-0.5cm}
\begin{colorblock}{white}{blue!30!black}{black}{blue!15!white}{}
\begin{itemize}
\item Map jobs onto the node-time space
\begin{itemize}
\item Assuming CPU time is the only resource
\end{itemize}
\item Need to find a balance between
\begin{itemize}
\item Honoring the order in which jobs are received
\item Maximizing resource utilization
\end{itemize}
\end{itemize}
\end{colorblock}
\end{columns}
\end{frame}

\begin{frame}
\frametitle{\small Backfilling}
\begin{columns}
\column{7cm}
\vspace{-1cm}
\begin{center}
\includegraphics[width=0.9\textwidth,clip=true]{JobSchedule-2}
\end{center}
\column{5cm}
\vspace{-0.5cm}
\begin{colorblock}{white}{blue!30!black}{black}{blue!15!white}{}
\begin{itemize}
\item A strategy to improve utilization
\begin{itemize}
\item Allow a job to jump ahead of others when there are enough idle nodes
\item Must not affect the estimated start time of the job with the highest priority
\end{itemize}
\item Enabled on all LONI and LSU HPC clusters
\end{itemize}
\end{colorblock}
\end{columns}
\end{frame}

\begin{frame}
\frametitle{\small How much time Should I request?}
\begin{columns}
\column{0.5\textwidth}
\vspace{-1cm}
\begin{center}
\includegraphics[width=0.9\textwidth,clip=true]{JobSchedule-3}
\end{center}
\column{0.5\textwidth}
\vspace{-1cm}
\begin{center}
\includegraphics[width=0.9\textwidth,clip=true]{JobSchedule-4}
\end{center}
\end{columns}
\begin{colorblock}{white}{blue!30!black}{black}{blue!15!white}{}
\begin{itemize}
\item Ask for an amount of time that is
\begin{itemize}
\item Long enough for your job to complete
\item As short as possible to increase the chance of backfilling
\end{itemize}
\end{itemize}
\end{colorblock}
\end{frame}

\begin{frame}
\frametitle{\small Job Queues}
\begin{colorblock}{white}{blue!30!black}{black}{blue!15!white}{}
\begin{itemize}
\item There are more than one job queue
\item Each job queue differs in
\begin{itemize}
\item Number of available nodes
\item Maximum run time
\item Maximum running jobs per user
\end{itemize}
\item The main purpose is to maximize utilization
\end{itemize}
\end{colorblock}
\end{frame}

\begin{frame}
\frametitle{\small Queue Characteristics: LONI Linux Clusters}
\begin{columns}
\column{12cm}
\vspace{-0.5cm}
\begin{colorblock}{white}{blue!30!black}{black}{blue!15!white}{QueenBee}
{\scriptsize
\begin{center}
\begin{tabular}{|m{0.1\textwidth}|m{0.1\textwidth}|m{0.1\textwidth}|m{0.12\textwidth}|m{0.1\textwidth}|m{0.25\textwidth}|}
\hline
Queue & Max Runtime & Total number of nodes & Max running jobs per user & Max nodes per job & Use \\
\hline
workq & \multirow{4}{*}{2 days} & 530 & \multirow{2}{*}{8} & 128 & Unpreemptable \\
\cline{1-1}\cline{3-3}\cline{5-6}
checkpt & & 668 & & 256 & preemptable\\
\cline{1-1}\cline{3-6}
preempt & & 668 & \multicolumn{2}{c|}{NA} & Requires permission \\
\cline{1-1}\cline{3-6}
priority & & 668 & \multicolumn{2}{c|}{NA} & Requires permission \\\hline 
\end{tabular}
\end{center}
}
\end{colorblock}
\begin{colorblock}{white}{blue!30!black}{black}{blue!15!white}{Other Clusters}
{\scriptsize
\begin{center}
\begin{tabular}{|m{0.1\textwidth}|m{0.1\textwidth}|m{0.1\textwidth}|m{0.12\textwidth}|m{0.1\textwidth}|m{0.25\textwidth}|}
\hline
Queue & Max Runtime & Total number of nodes & Max running jobs per user & Max nodes per job & Use \\
\hline
single & 14 days & 16 & 64 & 1 & Single processor jobs \\\hline
workq & \multirow{4}{*}{3 days} & 96 & \multirow{2}{*}{8} & 40 & Unpreemptable \\
\cline{1-1}\cline{3-3}\cline{5-6}
checkpt & & 128 & & 64 & preemptable\\
\cline{1-1}\cline{3-6}
preempt & & 64 & \multicolumn{2}{c|}{NA} & Requires permission \\
\cline{1-1}\cline{3-6}
priority & & 64 & \multicolumn{2}{c|}{NA} & Requires permission\\
\hline
\end{tabular}
\end{center}
}
\end{colorblock}
\end{columns}
\end{frame}

\begin{frame}
\frametitle{\small Queue Characteristics: LSU HPC Linux Clusters}
\begin{columns}
\column{12cm}
\vspace{-0.5cm}
\begin{colorblock}{white}{blue!30!black}{black}{blue!15!white}{Tezpur}
{\scriptsize
\begin{center}
\begin{tabular}{|m{0.1\textwidth}|m{0.1\textwidth}|m{0.1\textwidth}|m{0.12\textwidth}|m{0.1\textwidth}|m{0.25\textwidth}|}
\hline
Queue & Max Runtime & Total number of nodes & Max running jobs per user & Max nodes per job & Use \\
\hline
single & \multirow{5}{*}{3 days} & 16 & 64 & 1 & Single processor jobs \\
\cline{1-1}\cline{3-6}
workq &  & 180 & \multirow{2}{*}{8} & 90 & Unpreemptable \\
\cline{1-1}\cline{3-3}\cline{5-6}
checkpt & & 344 & & 180 & preemptable\\
\cline{1-1}\cline{3-6}
preempt & &  \multicolumn{3}{c|}{NA} & Requires permission \\
\cline{1-1}\cline{3-6}
priority & &  \multicolumn{3}{c|}{NA} & Requires permission \\\hline 
\end{tabular}
\end{center}
}
\end{colorblock}
\begin{colorblock}{white}{blue!30!black}{black}{blue!15!white}{Philip}
{\scriptsize
\begin{center}
\begin{tabular}{|m{0.1\textwidth}|m{0.1\textwidth}|m{0.1\textwidth}|m{0.12\textwidth}|m{0.1\textwidth}|m{0.25\textwidth}|}
\hline
Queue & Max Runtime & Total number of nodes & Max running jobs per user & Max nodes per job & Use \\
\hline
single & \multirow{6}{*}{3 days} & 24 & \multirow{4}{*}{12} & 1 & Single processor jobs \\
\cline{1-1}\cline{3-3}\cline{5-6}
workq &  & 28 &  & \multirow{3}{*}{5} & Unpreemptable \\
\cline{1-1}\cline{3-3}\cline{6-6}
checkpt & & 28 & &  & preemptable\\
\cline{1-1}\cline{3-3}\cline{6-6}
bigmem & & 5 & & \\
\cline{1-1}\cline{3-6}
preempt & &  \multicolumn{3}{c|}{NA} & Requires permission \\
\cline{1-1}\cline{3-6}
priority & &  \multicolumn{3}{c|}{NA} & Requires permission \\\hline 
\end{tabular}
\end{center}
}
\end{colorblock}
\end{columns}
\end{frame}

\begin{frame}
\frametitle{\small Queue Characteristics: LSU HPC AIX Clusters}
\begin{columns}
\column{12cm}
\vspace{-0.5cm}
\begin{colorblock}{white}{blue!30!black}{black}{blue!15!white}{Pelican}
{\scriptsize
\begin{center}
\begin{tabular}{|m{0.1\textwidth}|m{0.1\textwidth}|m{0.1\textwidth}|m{0.12\textwidth}|m{0.1\textwidth}|m{0.25\textwidth}|}
\hline
Queue & Max Runtime & Total number of processors & Max running jobs per user & Max processors per job & Use \\
\hline
SP5L &  4 hours & \multirow{3}{*}{256} & \multirow{2}{*}{8} & 256 & Short Jobs \\
\cline{1-2}\cline{5-6}
MP5L & 5 days &  & & 128 & Medium Jobs\\
\cline{1-2}\cline{4-6}
LP5L & 7 days &  &  4 & 64 & Long Jobs  \\
\hline
\end{tabular}
\end{center}
}
\end{colorblock}
\begin{colorblock}{white}{blue!30!black}{black}{blue!15!white}{Pandora}
{\scriptsize
\begin{center}
\begin{tabular}{|m{0.1\textwidth}|m{0.1\textwidth}|m{0.1\textwidth}|m{0.12\textwidth}|m{0.1\textwidth}|m{0.25\textwidth}|}
\hline
Queue & Max Runtime & Total number of processors & Max running jobs per user & Max processors per job & Use \\
\hline
interactive &  30mins & 8 & \multirow{3}{*}{6} & 8 & Interactive Jobs \\
\cline{1-3}\cline{5-6}
workq & 3 days & 224 & & 128 & Standard Queue\\
\cline{1-3}\cline{5-6}
single & 7 days & 64  &   & 32 & Single Node Jobs  \\
\hline
\end{tabular}
\end{center}
}
\end{colorblock}
\end{columns}
\end{frame}

\subsection{Job Manager Commands}
\begin{frame}
\frametitle{\small Basic Job Manager Commands}
\begin{colorblock}{white}{blue!30!black}{black}{blue!15!white}{}
\begin{itemize}
\item Queue querying
\begin{itemize}
\item Check how busy the cluster is
\end{itemize}
\item Job submission
\item Job monitoring
\begin{itemize}
\item Check job status (estimated start time, remaining run time, etc)
\end{itemize}
\item Job manipulation
\begin{itemize}
\item Cancel/Hold jobs
\end{itemize}
\end{itemize}
\end{colorblock}
\end{frame}

\begin{frame}[fragile]
\frametitle{\small Queue Querying: Linux Clusters}
%\begin{colorblock}{white}{blue!30!black}{black}{blue!15!white}{}
\begin{itemize}
\item \texttt{qfree}: show number of free,busy and queued nodes
\item \texttt{qfreeloni}: run \texttt{qfree} on all LONI Linux clusters
\end{itemize}
%\end{colorblock}
\begin{colorblock}{white}{blue!30!black}{black}{green!15!white}{}
{\tiny
\begin{verbatim}
[apacheco@eric2 ~]$ qfree
PBS total nodes: 128,  free: 49,  busy: 79,  down: 0,  use: 61\%
PBS workq nodes: 96,  free: 40,  busy: 28,  queued: 0
PBS checkpt nodes: 104,  free: 40,  busy: 35,  queued: 0
PBS single nodes: 32,  free: 9 *36,  busy: 16,  queued: 366
[apacheco@eric2 ~]$ qfreeloni
-------- qb --------
PBS total nodes: 668,  free: 3,  busy: 647,  down: 18,  use: 96\%
PBS workq nodes: 530,  free: 0,  busy: 278,  queued: 367
PBS checkpt nodes: 668,  free: 1,  busy: 369,  queued: 770
-------- eric --------
PBS total nodes: 128,  free: 49,  busy: 79,  down: 0,  use: 61\%
PBS workq nodes: 96,  free: 40,  busy: 28,  queued: 0
PBS checkpt nodes: 104,  free: 40,  busy: 35,  queued: 0
PBS single nodes: 32,  free: 9 *36,  busy: 16,  queued: 366
-------- louie --------
PBS total nodes: 128,  free: 44,  busy: 83 *2,  down: 1,  use: 64\%
PBS workq nodes: 104,  free: 40,  busy: 0,  queued: 0
PBS checkpt nodes: 128,  free: 44,  busy: 82,  queued: 50
PBS single nodes: 32,  free: 7 *26,  busy: 2,  queued: 0
-------- oliver --------
PBS total nodes: 128,  free: 74,  busy: 52,  down: 2,  use: 40\%
PBS workq nodes: 62,  free: 8,  busy: 11,  queued: 0
...
\end{verbatim}
}
\end{colorblock}
\end{frame}

\begin{frame}[fragile]
\frametitle{\small Queue Querying: AIX Clusters}
\begin{itemize}
\item Command: \texttt{llclass}
\end{itemize}
\begin{columns}
\column{12cm}
\begin{colorblock}{white}{blue!30!black}{black}{green!15!white}{}
{\tiny
\begin{verbatim}
apacheco@peg304$ llclass
Name                 MaxJobCPU     MaxProcCPU  Free   Max  Description
                    d+hh:mm:ss     d+hh:mm:ss Slots Slots
--------------- -------------- -------------- ----- -----  ---------------------
interactive          undefined      undefined     4     4  Interactive Parallel jobs running on interactive node
SP5L                 unlimited      unlimited   154   256  Short (4 hours) Parallel queue on Power5+ machines with Large memory (max_processors 256)
MP5L                 unlimited      unlimited   154   256  Middle (5 days) Parallel queue on Power5+ machines with Large memory (max_processors 128)
LP5L                 unlimited      unlimited   154   256  Long (7 days) Parallel queue on Power5+ machines with Large memory (max_processors 64)
--------------------------------------------------------------------------------
"Free Slots" values of the classes "SP5L", "MP5L", "LP5L" are constrained by the MAX_STARTERS limit(s).
pandora1:~ apacheco$ llclass
Name                 MaxJobCPU     MaxProcCPU  Free   Max Description          
                    d+hh:mm:ss     d+hh:mm:ss Slots Slots                      
--------------- -------------- -------------- ----- ----- ---------------------
interactive          unlimited      unlimited     8     8 Queue for interactive jobs; maximum runtime of 30 minutes.
workq                unlimited      unlimited    80   224 Standard queue for job submissions; maximum runtime of 3 days.
single               unlimited      unlimited    32    64 Queue for single-node job submissions; maximum runtime of 7 days.
--------------------------------------------------------------------------------
"Free Slots" values of the classes "workq", "single" are constrained by the MAX_STARTERS limit(s).

\end{verbatim}
}
\end{colorblock}
\end{columns}
\end{frame}

\subsection{Job Types}
\begin{frame}[overlayarea]
\frametitle{\small Job Types}
\only<1>{
\begin{colorblock}{white}{blue!30!black}{black}{blue!15!white}{Interactive Jobs}
\begin{itemize}
\item Set up an interactive environment on compute nodes for users
\begin{itemize}
\item[$\blacksquare$] Advantage: can run programs interactively
\item[$\blacksquare$] Disadvantage: must be present when job starts
\end{itemize}
\item Purpose:  testing and debugging code. \textbf{Do not run jobs on head node!!!}
\item[] \texttt{qsub -I -V -l walltime=<hh:mm:ss>,nodes=<\# of nodes>:ppn=cpu -A <your allocation> -q <queue name>}
\item On QueenBee, cpu=8
\item Other LONI Clusters:  cpu=4 (parallel jobs) or cpu=1 (single queue)
\item To enable X-forwarding: add \texttt{-X}
\end{itemize}
\end{colorblock}
}
\only<2>{
\begin{colorblock}{white}{blue!30!black}{black}{blue!15!white}{Batch Jobs}
\begin{itemize}
\item Executed using a batch script without user intervention
\begin{itemize}
\item[$\blacksquare$] Advantage: system takes care of running the job
\item[$\blacksquare$] Disadvantage: can change sequence of commands after submission
\end{itemize}
\item Useful for Production runs
\item[] \texttt{qsub <job script>}
\item[] \texttt{llsubmit <job script>}
\end{itemize}
\end{colorblock}
}
\end{frame}

\subsection{Job Submission Scripts}
\begin{frame}[fragile]
\frametitle{\small PBS Job Script: Parallel Jobs}
\begin{columns}
\column{7.2cm}
\begin{colorblock}{white}{blue!30!black}{black}{green!15!white}{}
{\footnotesize
\begin{verbatim}
#!/bin/bash
#PBS -l nodes=4:ppn=4
#PBS -l walltime=24:00:00
#PBS -N myjob
#PBS -o <file name>
#PBS -e <file name>
#PBS -q checkpt	
#PBS -A <loni_allocation>
#PBS -m e
#PBS -M <email address>

<shell commands>			
mpirun  -machinefile $PBS_NODEFILE \
 -np 16 <path_to_executable> <options>
<shell commands>
\end{verbatim}
}
\end{colorblock}
\column{4.5cm}
\begin{colorblock}{white}{blue!30!black}{black}{red!15!white}{}
{\footnotesize %\color{red}
\begin{verbatim}
Shell being used
# of nodes & processors
Maximum walltime
Job name
standard output
standard error
Queue name
Allocation name
Send mail when job ends 
to this address

shell commands
run parallel job

shell commands
\end{verbatim}
}
\end{colorblock}
\end{columns}
\end{frame}

\begin{frame}[fragile]
\frametitle{\small PBS Job Script: Serial Jobs}
\begin{columns}
\column{7.2cm}
\begin{colorblock}{white}{blue!30!black}{black}{green!15!white}{}
{\footnotesize
\begin{verbatim}
#!/bin/bash
#PBS -l nodes=1:ppn=1
#PBS -l walltime=24:00:00
#PBS -N myjob
#PBS -o <file name>
#PBS -e <file name>
#PBS -q single	
#PBS -A <loni_allocation>
#PBS -m e
#PBS -M <email address>

<shell commands>			
<path_to_executable> <options>
<shell commands>
\end{verbatim}
}
\end{colorblock}
\column{4.5cm}
\begin{colorblock}{white}{blue!30!black}{black}{red!15!white}{}
{\footnotesize %\color{red}
\begin{verbatim}
Shell being used
# of nodes & processors
Maximum walltime
Job name
standard output
standard error
Use single queue 
Allocation name
Send mail when job ends 
to this address

shell commands
run parallel job
shell commands
\end{verbatim}
}
\end{colorblock}
\end{columns}
\end{frame}



\begin{frame}[fragile,allowframebreaks]
\frametitle{\small Loadleveler Job Script}
\vspace{-0.5cm}
\begin{columns}
\column{7.2cm}
\begin{colorblock}{white}{blue!30!black}{black}{green!15!white}{}
{\footnotesize
\begin{verbatim}
#!/bin/sh
#@ job_type = parallel 
#@ output = $(jobid).out
#@ error = $(jobid).err
#@ notification = error
#@ notify_user = youremail@domain
#@ class = checkpt
#@ wall_clock_limit = 24:00:00
#@ node_usage = shared
#@ node = 2,2
#@ total_tasks = 16
#@ requirements = (Arch == "POWER5")
#@ environment = COPY_ALL
#@ queue

<shell commands>			
poe <path_to_executable> <options>
<shell commands>
\end{verbatim}
}
\end{colorblock}
\column{4.5cm}
\begin{colorblock}{white}{blue!30!black}{black}{red!15!white}{}
{\footnotesize %\color{red}
\begin{verbatim}
Shell being used
Job Type
standard output
standard error
notify on error
to mail address
job queue
max walltime
node usage
# of nodes
total processors
job requirements
environment


shell commands
run parallel job
shell commands
\end{verbatim}
}
\end{colorblock}
\end{columns}
\newpage
\vspace{-0.5cm}
\begin{columns}
\column{7.2cm}
\begin{colorblock}{white}{blue!30!black}{black}{green!15!white}{}
{\footnotesize
\begin{verbatim}
#!/bin/sh
#@ job_type = serial 
#@ output = $(jobid).out
#@ error = $(jobid).err
#@ notification = error
#@ notify_user = youremail@domain
#@ class = checkpt
#@ wall_clock_limit = 24:00:00
#@ node_usage = shared
#@ node = 1
#@ total_tasks = 1
#@ requirements = (Arch == "POWER5")
#@ environment = COPY_ALL
#@ queue

<shell commands>			
<path_to_executable> <options>
<shell commands>
\end{verbatim}
}
\end{colorblock}
\column{4.5cm}
\begin{colorblock}{white}{blue!30!black}{black}{red!15!white}{}
{\footnotesize %\color{red}
\begin{verbatim}
Shell being used
Job Type
standard output
standard error
notify on error
to mail address
job queue
max walltime
node usage
# of nodes
total processors
job requirements
environment


shell commands
run parallel job
shell commands
\end{verbatim}
}
\end{colorblock}
\end{columns}
\newpage
\begin{colorblock}{white}{blue!30!black}{black}{blue!15!white}{}
\begin{itemize}
\item On Pandora: 
\begin{itemize}
\item \texttt{\#@ resources = ConsumableMem(512 mb) ConsumableCPUS(1)} is required
\item \texttt{\#@ requirements = (Arch == "POWER7")}
\item \texttt{\#@ network.MPI\_LAPI = sn\_all,shared,US,HIGH}
\end{itemize}
\end{itemize}
\end{colorblock}
\end{frame}

\begin{frame}
 \begin{eblock}{Exercise 4: Job Submission}
  \begin{itemize}
   \item Write a job submission script to execute the \texttt{hello\_mpi} program.
   \item Submit the script to the job manager.
  \end{itemize}
 \end{eblock}

\end{frame}


\subsection{Job Monitoring \& Manipulation}
\begin{frame}
\frametitle{\small Job Monitoring}
\begin{colorblock}{white}{blue!30!black}{black}{blue!15!white}{Linux Clusters}
\begin{itemize}
\item \texttt{showstart <job id>}
\begin{enumerate}
\item[$\vardiamond$] Check estimated time when job can start
\end{enumerate}
\item When can the estimated time change
\begin{enumerate}
\item[$\vardiamond$] Higher priority job gets submitted
\item[$\vardiamond$] Running jobs terminate earlier than time requested
\item[$\vardiamond$] System has trouble starting your job
\end{enumerate}
\item \texttt{qstat <options> <job id>}
\begin{enumerate}
\item[$\vardiamond$] Show information on job status
\item[$\vardiamond$] All jobs displayed if \texttt{<job id>} is omitted
\item[$\vardiamond$] \texttt{qstat -u <username>}: Show jobs belonging to \texttt{<username>}
\item[$\vardiamond$] \texttt{qstat -a <job id>}: Displat in an alternative format
\end{enumerate}
\item \texttt{qshow <job id>}
\begin{enumerate}
\item[$\vardiamond$] Show information of running job \texttt{<job id>}: node running on and CPU load
\end{enumerate}
\end{itemize}
\end{colorblock}
\end{frame}

\begin{frame}[fragile]
\frametitle{\small Job Monitoring}
\begin{colorblock}{white}{blue!30!black}{black}{blue!15!white}{AIX Clusters}
\begin{itemize}
\item \texttt{llq <options> <job id>}
\begin{itemize}
\item[$\vardiamond$] All jobs are displayed if \texttt{<job id>} is omitted
\item[$\vardiamond$] Display detailed information: \texttt{llq -l <job id>}
\item[$\vardiamond$] Check estimated start time: \texttt{llq -s <job id>}
\item[$\vardiamond$] Show jobs from a specific user: \texttt{llq -u <username>}
\end{itemize}
\end{itemize}
\end{colorblock}
{\tiny
\begin{alertblock}{}
\begin{verbatim}
apacheco@l3f1n03$ llq
Id                       Owner      Submitted   ST PRI Class        Running on 
------------------------ ---------- ----------- -- --- ------------ -----------
l3f1n03.14904.0          huiwu       7/16 15:45 R  50  checkpt      l3f1n09    
l3f1n03.14908.0          srick       7/18 10:15 R  50  checkpt      l3f1n13    
l3f1n03.14909.0          srick       7/18 10:18 R  50  checkpt      l3f1n04    
l3f1n03.14911.0          huiwu       7/19 13:48 R  50  checkpt      l3f1n11    
l3f1n03.14910.0          srick       7/18 10:18 R  50  checkpt      l3f1n06    

5 job step(s) in queue, 0 waiting, 0 pending, 5 running, 0 held, 0 preempted
\end{verbatim}
\end{alertblock}
}
\end{frame}

\begin{frame}[fragile]
\frametitle{\small Job Monitoring}
\begin{colorblock}{white}{blue!30!black}{black}{blue!15!white}{AIX Clusters}
\begin{itemize}
\item \texttt{showllstatus.py}: Show job status and node running on
\end{itemize}
\end{colorblock}
\begin{alertblock}{}
{\tiny
\begin{verbatim}
apacheco@peg304$ showllstatus.py 
Node     Status   Load       Arch Node     Status   Load       Arch
ian1     Idle     1.02     Power4 pen09    Busy     16.28    Power5
pen01    Run      4.08     Power5 pen10    Busy     16.33    Power5
pen02    Run      2.01     Power5 pen11    Idle     0.00     Power5
pen03    Run      4.50     Power5 pen12    Idle     0.00     Power5
pen04    Run      7.04     Power5 pen13    Busy     16.21    Power5
pen05    Run      3.99     Power5 pen14    Run      2.50     Power5
pen06    Busy     16.30    Power5 pen15    Idle     0.00     Power5
pen07    Run      2.00     Power5 pen16    Idle     0.00     Power5
pen08    Run      4.07     Power5 

Step ID         Owner        Status   Class    Hosts  Queue Date   Disp. Date 
ian1.97678.0    nserno          R     MP5L       1    07/19 11:19  07/19 11:19
ian1.97677.0    nserno          R     MP5L       1    07/19 11:16  07/19 11:16
ian1.97672.0    cmcfer1         R     MP5L       1    07/19 08:38  07/19 08:38
ian1.97650.0    nserno          R     MP5L       1    07/18 13:30  07/18 13:30
ian1.97647.0    yuzhiyi         R     MP5L       4    07/18 10:27  07/18 10:27
ian1.97646.0    jgibs22         R     MP5L       1    07/18 10:09  07/18 10:09
ian1.97645.0    nserno          R     MP5L       1    07/17 13:20  07/17 21:40
ian1.97644.0    nserno          R     MP5L       1    07/17 13:20  07/17 17:20
ian1.97643.0    nserno          R     MP5L       1    07/17 13:20  07/17 16:51
\end{verbatim}
}
\end{alertblock}
\end{frame}

\begin{frame}
\frametitle{\small Job Manipulation}
\begin{colorblock}{white}{blue!30!black}{black}{blue!15!white}{Linux Clusters}
\begin{itemize}
\item \texttt{qdel <job id>}
\begin{enumerate}
\item[$\vardiamond$] Cancel a running or queued job
\end{enumerate}
\item \texttt{qhold <job id>}
\begin{enumerate}
\item[$\vardiamond$] Put a queued job on hold
\end{enumerate}
\item \texttt{qrls <job id>}
\begin{enumerate}
\item[$\vardiamond$] Resume a held job
\end{enumerate}
\end{itemize}
\end{colorblock}
\begin{colorblock}{white}{blue!30!black}{black}{blue!15!white}{AIX Clusters}
\begin{itemize}
\item \texttt{llcancel <job id>}
\begin{enumerate}
\item[$\vardiamond$] Cancel a running or queued job
\end{enumerate}
\item \texttt{llhold <job id>}
\begin{enumerate}
\item[$\vardiamond$] Put a queued job on hold
\end{enumerate}
\item \texttt{llhold -r <job id>}
\begin{enumerate}
\item[$\vardiamond$] Resume a held job
\end{enumerate}
\end{itemize}
\end{colorblock}
\end{frame}

\section{HPC Help}
\begin{frame}
\frametitle{\small Additional Help}
\begin{itemize}
\item User's Guide
\begin{enumerate}
\item[$\vardiamond$]HPC: \url{http://www.hpc.lsu.edu/help}
\item[$\vardiamond$]LONI: \url{https://docs.loni.org/wiki/Main_Page}
\end{enumerate}
\item Contact us
\begin{enumerate}
\item[$\vardiamond$]Email ticket system: sys-help@loni.org
\item[$\vardiamond$]Telephone Help Desk: 225-578-0900
\item[$\vardiamond$]Walk-in consulting session at Middleton Library
\begin{enumerate}
\item[$\bigstar$]Tuesdays and Thursdays only
\end{enumerate}
\item[$\vardiamond$]Instant Messenger (AIM, Yahoo Messenger, Google Talk)
\begin{enumerate}
\item[$\bigstar$]Add "lsuhpchelp"
\end{enumerate}
\end{enumerate}
\end{itemize}
\end{frame}
\end{document}

