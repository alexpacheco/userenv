\documentclass[slidestop,mathserif,compress,xcolor=svgnames,table]{beamer} 
\mode<presentation>
{  
  \setbeamertemplate{background canvas}[vertical shading][bottom=blue!5,top=blue!5]
  \setbeamertemplate{navigation symbols}{}%{\insertsectionnavigationsymbol}
  \usetheme{LSU}
}

\usepackage{pgf,pgfarrows,pgfnodes,pgfautomata,pgfheaps,pgfshade}
\usepackage{amsmath,amssymb,amsfonts}
\usepackage{multirow}
\usepackage{tabularx}
\usepackage{booktabs}
\usepackage{colortbl}
\usepackage{tikz}
\usetikzlibrary{shapes,arrows}
\usetikzlibrary{calc}
\pgfdeclarelayer{background}
\pgfdeclarelayer{foreground}
\pgfsetlayers{background,main,foreground}
\usepackage[latin1]{inputenc}
\usepackage{colortbl}
\usepackage[english]{babel}
\usepackage{hyperref}
\usepackage{movie15}
\hypersetup{
  pdftitle={LONI Programming Environment},
  pdfauthor={Alexander B. Pacheco, User Services Consultant, Louisiana State University}
}
%\usepackage{movie15}
\usepackage{times}

\setbeamercovered{dynamic}
\beamersetaveragebackground{DarkBlue!2}
\beamertemplateballitem

\usepackage[english]{babel}
\usepackage[latin1]{inputenc}
\usepackage{times}
\usepackage{amsmath}
\usepackage[T1]{fontenc}
\usepackage{graphicx}
\definecolor{DarkGreen}{rgb}{0.0,0.3,0.0}
\definecolor{Blue}{rgb}{0.0,0.0,0.8} 
\definecolor{dodgerblue}{rgb}{0.1,0.1,1.0}
\definecolor{indigo}{rgb}{0.41,0.1,0.0}
\definecolor{seagreen}{rgb}{0.1,1.0,0.1}
\DeclareSymbolFont{extraup}{U}{zavm}{m}{n}
\DeclareMathSymbol{\vardiamond}{\mathalpha}{extraup}{87}
\newcommand*\up{\textcolor{green}{%
  \ensuremath{\blacktriangle}}}
\newcommand*\down{\textcolor{red}{%
  \ensuremath{\blacktriangledown}}}
\newcommand*\const{\textcolor{darkgray}%
  {\textbf{--}}}


\setbeamercolor{uppercol}{fg=white,bg=red!30!black}%
\setbeamercolor{lowercol}{fg=black,bg=red!15!white}%
\setbeamercolor{uppercol1}{fg=white,bg=blue!30!black}%
\setbeamercolor{lowercol1}{fg=black,bg=blue!15!white}%%
\setbeamercolor{uppercol2}{fg=white,bg=green!30!black}%
\setbeamercolor{lowercol2}{fg=black,bg=green!15!white}%
\newenvironment{colorblock}[4]
{
\setbeamercolor{upperblock}{fg=#1,bg=#2}
\setbeamercolor{lowerblock}{fg=#3,bg=#4}
\begin{beamerboxesrounded}[upper=upperblock,lower=lowerblock,shadow=true]}
{\end{beamerboxesrounded}}
\newenvironment{ablock}[0]
{
\begin{beamerboxesrounded}[upper=uppercol,lower=lowercol,shadow=true]}
{\end{beamerboxesrounded}}
\newenvironment{bblock}[0]
{
\begin{beamerboxesrounded}[upper=uppercol1,lower=lowercol1,shadow=true]}
{\end{beamerboxesrounded}}
\newenvironment{eblock}[0]
{
\begin{beamerboxesrounded}[upper=uppercol2,lower=lowercol2,shadow=true]}
{\end{beamerboxesrounded}}


\title{LONI Programming Environment}


\author[Alex Pacheco]{\large{Alexander~B.~Pacheco}}
       
%\institute[High Performance Computing @ Louisiana State University - http://www.hpc.lsu.edu] {\inst{}\footnotesize{User Services Consultant\\LSU HPC \& LONI\\sys-help@loni.org}}
\institute[LONI HPC Workshop, University of Louisiana - Lafayette, 11/09/2011] {\inst{}\footnotesize{User Services Consultant\\LSU HPC \& LONI\\sys-help@loni.org}}

\date[\hfill{Nov 9, 2011}]{\scriptsize{LONI HPC Workshop\\University of Louisiana - Lafayette\\Lafayette\\Nov 9, 2011}}
     
\subject{Talks}
\keywords{LONI Computing Resources, User Environment, Job Management, PBS, Loadleveler}
% This is only inserted into the PDF information catalog. Can be left
% out. 




% If you have a file called "university-logo-filename.xxx", where xxx
% is a graphic format that can be processed by latex or pdflatex,
% resp., then you can add a logo as follows:

% Main Logo on bottom left
\pgfdeclareimage[height=0.55cm]{its-logo}{PUR_BLK_HOR}
\logo{\pgfuseimage{its-logo}}
% University Logo on top left
\pgfdeclareimage[height=0.55cm]{university-logo}{LSUGeauxPurp}
\tllogo{\pgfuseimage{university-logo}}
% Logo at top right
\pgfdeclareimage[height=0.6cm]{institute-logo}{LONI}
\trlogo{\pgfuseimage{institute-logo}}
% Logo at bottom right
\pgfdeclareimage[height=0.55cm]{hpc-logo}{LONI-2}
\brlogo{\pgfuseimage{hpc-logo}}

% Delete this, if you do not want the table of contents to pop up at
% the beginning of each subsection:
 \AtBeginSection[]
 {
   \begin{frame}<beamer>
    \frametitle{\small{Outline}}
     \small
     \tableofcontents[currentsection,currentsubsection]
   \end{frame}
 }

\begin{document}

\frame{\titlepage}

\normalsize
\begin{frame}[label=toc,squeeze]
  \footnotesize
  \frametitle{\small{Outline}}
  \tableofcontents
\end{frame}

\normalsize

%\part{Introduction}
\section{Hardware Overview}

\begin{frame}
  \frametitle{\small LONI \& LSU HPC Clusters}
  \begin{itemize}
    \item Two major architectures.
  \end{itemize}
  \begin{columns}
    \column{0.5\textwidth}
    \begin{colorblock}{white}{blue!30!black}{black}{blue!15!white}{\bf Linux Clusters}
      \begin{itemize}
	\item Vendor: Dell
	\item OS: Red Hat 
	\item CPU: Intel Xeon
      \end{itemize}
    \end{colorblock}
    \column{0.5\textwidth}
    \begin{colorblock}{white}{blue!30!black}{black}{blue!15!white}{\bf AIX Clusters}
      \begin{itemize}
	\item Vendor: IBM
	\item OS: AIX 
	\item CPU: Power 5
      \end{itemize}
    \end{colorblock}
  \end{columns}
  \vspace{1cm}
  \begin{itemize}
    \item The LONI AIX clusters are on a path to decommissioning.
  \end{itemize}
\end{frame}

\begin{frame}
  \scriptsize{
  \vspace{-0.2cm}
  \begin{block}{Linux Clusters}
    \begin{center}
      \begin{tabular}{|c|c|c|c|c|c|}
	\hline
	& Name & Peak TeraFLOPS/s & Location & Status & Login\\
	\hline
	\multirow{6}{*}{LONI} & QueenBee & 50.7 & ISB & Production & LONI \\
                              & Eric & 4.7 & LSU & Production & LONI\\
                              & Louie & 4.7 & Tulane & Production & LONI\\
                              & Oliver & 4.7 & ULL & Production & LONI\\
                              & Painter & 4.7 & LaTech & Production & LONI\\
                              & Poseidon & 4.7 & UNO & Production & LONI\\
	\hline
      \end{tabular}
    \end{center}
  \end{block}
  \begin{block}{AIX Clusters}
    \begin{center}
      \def\firstrowcolor{\rowcolor{green}}
      \def\secondrowcolor{\rowcolor{blue!50}}
      \def\thirdrowcolor{\rowcolor{tigerspurple!80}}
      \begin{tabular}{|c|c|c|c|c|c|}
	\hline
	& Name & Peak TF/s & Location & Status & Login\\
	\hline
	\multirow{5}{*}{LONI} & Bluedawg & 0.85 & LaTech & Production & LONI \\
                              & Ducky & 0.85 & UNO & Decommissioned & LONI\\
                              & Lacumba & 0.85 & Southern & Decommissioned & LONI\\
                              & Neptune & 0.85 & Tulane & Decommissioned & LONI\\
                              & Zeke & 0.85 & ULL & Decommissioned & LONI\\
	\hline
      \end{tabular}
    \end{center}
  \end{block}
  }
\end{frame}


\begin{frame}
  \frametitle{\small Account Management}
  \begin{bblock}{Getting an Account}
    \begin{itemize}
      \item LONI account
      \item[] \url{https://allocations.loni.org}
%       \item LSU HPC account
%       \item[] \url{https://accounts.hpc.lsu.edu}
      \item Request Allocations
      \item[] \url{https://allocations.loni.org}
      \item All LONI AIX clusters are being decommissioned.
%       \item Newest cluster at LSU HPC is Pandora.
    \end{itemize}
  \end{bblock}

  \begin{eblock}{Login Shell}
    \begin{itemize}
      \item The default Login shell is bash
      \item Supported Shells: bash, tcsh, ksh, csh \& sh
      \item Change Login Shell at the profile page
    \end{itemize}
   \end{eblock}
\end{frame}

\begin{frame}
  \frametitle{\small Cluster Architecture}
  \begin{columns}
    \column{4cm}
    \begin{itemize}
      \item A cluster is a group of computers (nodes) that works together closely
      \item Type of nodes
      \begin{enumerate}
	{\scriptsize
	\item[$\vardiamond$] Head node
	\item[$\vardiamond$] Compute node
	}
      \end{enumerate}
    \end{itemize}
    \column{7cm}
    \begin{center}
      \includegraphics[width=1.05\textwidth,clip=true]{cluster}
    \end{center}
  \end{columns}
\end{frame}

\begin{frame}
  \frametitle{\small Cluster Hardware}
  \begin{itemize}
    \item Queen Bee 
    \begin{enumerate}
      {\scriptsize
      \item[$\vardiamond$]668 nodes: 8 Intel Xeon cores @ 2.33 GHz
      \item[$\vardiamond$]8 GB RAM
      \item[$\vardiamond$]192 TB storage
      }
    \end{enumerate}
    \item Other LONI Linux clusters
    \begin{enumerate}
      {\scriptsize
      \item[$\vardiamond$]128 nodes: 4 Intel Xeons cores @ 2.33 GHz
      \item[$\vardiamond$]4 GB RAM
      \item[$\vardiamond$]9 TB storage
      }
    \end{enumerate}
    \item LONI AIX clusters (All except Bluedawg decommissioned)
    \begin{enumerate}
      {\scriptsize
      \item[$\vardiamond$]14 Power5 nodes, 8 IBM Power5 processors @ 1.9 GHz per node
      \item[$\vardiamond$]16 GB RAM
      \item[$\vardiamond$]280 GB storage
      }
    \end{enumerate}
  \end{itemize}
\end{frame}

\begin{frame}
  \frametitle{\small Why is Cluster Hardware important?} 
  \begin{columns}
    \column{5cm}
    \begin{itemize}
      \item There are numerous different architectures in the HPC world.
      \item Choose the software to install or use depending on cluster architecture.
      \begin{enumerate}
	\item Linux: EM64T, AMD64, X86\_64
	\item AIX: Power5, Power7
      \end{enumerate}
    \end{itemize}
    \column{7cm}
    \vspace{-1cm}
    \begin{center}
      \includegraphics[width=\textwidth,clip=true]{namd-down}
    \end{center}
  \end{columns}
\end{frame}

\begin{frame}
  \frametitle{\small Usage: Max Memory}
  \begin{itemize}
    \item The amount of installed memory less the amount that is used by the operating system and other utilities 
    \item Max amount per node
    \begin{enumerate}
      {\scriptsize
      \item[$\vardiamond$]Linux clusters: $\sim$6 GB for Queen Bee, $\sim$3 GB for others
      \item[$\vardiamond$]AIX clusters: $\sim$13 GB 
      }
    \end{enumerate}
  \end{itemize}
\end{frame}


\section{User Environment}
\subsection{Accessing LONI HPC clusters}
\begin{frame}
  \frametitle{\small Accessing LONI clusters}
  \begin{itemize}
    \item LONI Host name: \texttt{<cluster name>.loni.org}
    \begin{itemize}
      \item[$\bigstar$] Eric: eric.loni.org
    \end{itemize}
    \item Use ssh to connect
    \begin{itemize}
      \item[$\bigstar$] $\ast$nix and Mac: \texttt{ssh <host name>}
      \item[$\bigstar$] Windows: use Putty, Secure Shell Client or Bitvise Tunnelier
    \end{itemize}
    \item The default Login shell is bash
    \item Supported shells: bash, tcsh, ksh, csh \& sh
    \item Change the login shell at the profile page
    \begin{enumerate}
      {\scriptsize
      \item[$\vardiamond$] LONI: \url{https://allocations.loni.org}
      }
    \end{enumerate}
    \item Reset your password
    \begin{enumerate}
      {\scriptsize
      \item[$\vardiamond$] LONI: \url{https://allocations.loni.org/user\_reset.php}
      }
    \end{enumerate}
  \end{itemize}
\end{frame}

\begin{frame}
  \frametitle{\small Connecting to Eric from a Linux box}
  \vspace{-0.5cm}
  \begin{columns}
    \column{12cm}
    \begin{center}
      \includegraphics[width=1.35\textheight,clip=true]{X11-Linux}
    \end{center}
  \end{columns}
\end{frame}

\begin{frame}
  \frametitle{\small Connecting to Eric from a Mac box}
  \vspace{-0.5cm}
  \begin{columns}
    \column{12cm}
    \begin{center}
      \includegraphics[width=1.35\textheight,clip=true]{X11-mac}
    \end{center}
  \end{columns}
\end{frame}

\begin{frame}
  \frametitle{\small Connecting to Eric from a Windows box}
  \begin{itemize}
    \item Download and Install
    \begin{enumerate}
      {\scriptsize
      \item X-Server: X-ming \url{http://www.straightrunning.com/XmingNotes/}
      \item SSH Client: Putty \url{http://www.chiark.greenend.org.uk/~sgtatham/putty/}
      \item SSH+SFTP/SCP Client: Bitvise Tunnelier \url{http://www.bitvise.com/tunnelier}
      }
    \end{enumerate}
  \end{itemize}
\end{frame}

\begin{frame}
  \frametitle{\small Start X-ming}
  \begin{center}
    \includegraphics[width=0.2\textwidth,clip=true]{X-ming-desk}
    \includegraphics[width=0.3\textwidth,clip=true]{X-ming-tray}
    \end{center}
\end{frame}

\begin{frame}
  \frametitle{\small Putty with X11}
  \begin{center}
    \only<1>{\includegraphics[width=0.5\textwidth,clip=true]{Putty-X11}}
    \only<2>{\includegraphics[width=0.5\textwidth,clip=true]{Putty-Login}}
    \only<3>{\includegraphics[width=0.75\textwidth,clip=true]{Putty-SSH-Key}}
    \only<4>{\includegraphics[width=0.75\textwidth,clip=true]{Putty-UID}}
    \only<5>{\includegraphics[width=0.75\textwidth,clip=true]{Putty-Loggedin}}
    \only<6>{\includegraphics[width=0.75\textheight,clip=true]{Putty-Jmol}}
  \end{center}
\end{frame}

\begin{frame}
  \frametitle{\small Configure Tunnelier/SSH Client to Tunnel X11 Connections}
  \begin{center}
    \only<1>{\includegraphics[width=0.75\textwidth,clip=true]{SSH-Secure-Shell}}
    \only<2>{\includegraphics[width=0.75\textheight,clip=true]{Tunnelier-X11}}
  \end{center}
\end{frame}

\subsection{File Systems}
\begin{frame}
  \frametitle{\small File Systems}
  \scriptsize{
  \begin{columns}
    \column{11cm}
    \begin{block}{}
      \begin{center}
	\begin{tabular}{|c|p{1.5cm}|c|c|p{2.5cm}|}
	  \hline
	  & Distributed File System & Throughput & File life time & Best used for \\
	  \hline
	  {Home} & Yes & Low & Unlimited & {Code in development, compiled executable}\\ 
	  \hline
	  Work & Yes & High & 30 days & Job input/output \\
	  \hline
	  {Local Scratch} & No & & Job Duration & Temporary files \\
	  \hline
	\end{tabular}
      \end{center}
    \end{block}
  \end{columns}
  }
  \begin{itemize}
    \item {\footnotesize Tips}
    \begin{enumerate}
      \item[$\vardiamond$] Never write job output to your home directory
      \item[$\vardiamond$] Do not write temporary files to /tmp, use local scratch or work space
      \item[$\vardiamond$] Work space is not for long term storage. Files are purged peridocally
      \item[$\vardiamond$] Use \texttt{rmpurge} to delete large amount of files.
    \end{enumerate}
  \end{itemize}
\end{frame}

\begin{frame}
  \frametitle{\small Disk Quota}
  {\scriptsize
  \begin{columns}
    \column{12cm}
    \begin{block}{}
      \begin{center}
	\begin{tabular}{|c|c|c|c|c|c|}
	  \hline
	  \multirow{2}{*}{Cluster} & \multicolumn{2}{c|}{Home} & \multicolumn{2}{c|}{Work} & Scratch \\
	  \cline{2-6}
	  & Access Point & Quota & Access Point & Quota & Access Point \\
	  \hline
	  LONI Linux & /home/\$USER & 5GB & /scratch/\$USER & 100GB & /var/scratch\\
	  \hline
	  LONI AIX & /home/\$USER & 500MB & /work/default/\$USER & 20GB & /var/scratch\\
	  \hline
% 	  HPC Linux & /home/\$USER & 5GB & /work/\$USER & NA & /var/scratch\\
% 	  \hline
% 	  HPC AIX & /home/\$USER & 5GB & /work/\$USER & 50GB & /scratch/local\\
% 	  \hline
	\end{tabular}
      \end{center}
    \end{block}
  \end{columns}
  }
  {\footnotesize
  \begin{itemize}
    \item No quota is enforced on the work space of QueenBee
    \item Work directory is created within an hour of first login
    \item Check current disk usage
    \item[] Linux: \texttt{showquota}
    \item[] AIX: \texttt{quota}
   \end{itemize}
  }
\end{frame}

\begin{frame}
  \frametitle{\small Exercise 1}
  \begin{eblock}{}
    \begin{itemize}
      \item Log in to any cluster
      \item Check your disk quota
      \begin{enumerate}
	\item Linux: \texttt{showquota}
	\item AIX: \texttt{quota}
      \end{enumerate}
      \item Copy the traininglab directory
      \item[] \texttt{cp -r /home/apacheco/traininglab .}
    \end{itemize}
  \end{eblock}

  \begin{ablock}{}
    \begin{itemize}
      \item If you are not familiar with working on a Linux/Unix system
      \begin{enumerate}
	\item Loni Moodle course @ \url{https://docs.loni.org/moodle}: HPC104 or HPC105
      \end{enumerate}
    \end{itemize}
  \end{ablock}
\end{frame}


\subsection{Software Management}
\begin{frame}
  \frametitle{\small Managing User Environment}
  \begin{itemize}
    \item Environment variables
    \begin{enumerate}
      \item[$\vardiamond$]PATH: where to look for executables
      \item[$\vardiamond$]LD\_LIBRARY\_PATH: where to look for shared libraries
      \item[$\vardiamond$]Other custom environment variables needed by various software
    \end{enumerate}
    \item {\bf SOFTENV} is a software that is used to set up these environment variables on all the clusters
    \begin{enumerate}
      \item[$\vardiamond$]More convenient than setting numerous environment variables in .bashrc or .cshrc
    \end{enumerate}
  \end{itemize}
\end{frame}

\begin{frame}[fragile]
  \frametitle{\small Listing All packages}
  \begin{itemize}
    \item Command \texttt{softenv} lists all packages that are managed by {\bf SOFTENV}
  \end{itemize}
  {\tiny
  \begin{alertblock}{}
    \begin{verbatim}
[apacheco@eric2 ~]$ softenv
SoftEnv version 1.6.2
...
----------------------------------------------------------------------------

These are the macros available:

*   @default                      


These are the keywords explicitly available:

    +ImageMagick-6.4.6.9-intel-11.1
                                   @types: Applications Visualization @name:
                                    ...
    +NAMD-2.6-intel-11.1-mvapich-1.1
                                   @types: Applications @name: NAMD @version:
                                     ...
    +NAMD-2.7b2-intel-11.1-mvapich-1.1
                                   @types: Applications @name: NAMD @version:
                                     ...
    \end{verbatim}
  \end{alertblock}
  }
\end{frame}

\begin{frame}[fragile]
  \frametitle{\small Searching for a Specific Package}
  \begin{itemize}
    \item Use \texttt{-k} option with \texttt{softenv}
  \end{itemize}
  {\tiny
  \begin{alertblock}{}
    \begin{verbatim}
[apacheco@eric2 ~]$ softenv -k gaussian
SoftEnv version 1.6.2
...
Search Regexp: gaussian
----------------------------------------------------------------------------

These are the macros available:

These are the keywords explicitly available:

    +gaussian-03                   @types: Applications Chemistry @name:
                                     Gaussian @version: 03 @build: @internal:
                                     ...
    +gaussian-09                   @types: Applications Chemistry @name:
                                     Gaussian @version: 09 @build: @internal:
                                     ...
    +gaussview-4.1.2               @types: Applications Chemistry @name:
                                     GaussView @version: 4.1.2 @build: - @about:
                                     ...

These are the keywords that are part of the software tree,
however, it is not suggested that you use these:
    \end{verbatim}
  \end{alertblock}
  }
\end{frame}

\begin{frame}[fragile]
  \frametitle{\footnotesize Setting up Environment via Softenv: One Time Change}
  \begin{itemize}
    \item Setting up environment variables to use a certain package in the current session only.
    \begin{enumerate}
      {\scriptsize
      \item[$\vardiamond$] Remove a package: \texttt{soft add <key>}
      \item[$\vardiamond$] Add a package: \texttt{soft add <key>}
      }
    \end{enumerate}
  \end{itemize}
  {\tiny
  \begin{alertblock}{}
    \begin{verbatim}
[apacheco@eric2 ~]$ which g09
/usr/local/packages/gaussian09/g09/g09
[apacheco@eric2 ~]$ soft delete +gaussian-09
[apacheco@eric2 ~]$ which g09
/usr/bin/which: no g09 in (/home/apacheco/bin:...
[apacheco@eric2 ~]$ soft add +gaussian-03
[apacheco@eric2 ~]$ which g03
/usr/local/packages/gaussian03/g03/g03
    \end{verbatim}
  \end{alertblock}
  }
\end{frame}

\begin{frame}[fragile]
  \frametitle{\footnotesize Setting up Environment via Softenv: Permanent Change}
  \begin{itemize}
    \item Setting up the environment variables to use a certain software package(s).
    \begin{enumerate}
      {\footnotesize
      \item[$\vardiamond$] First add the key to \texttt{\$HOME/.soft}.
      \item[$\vardiamond$] Execute \texttt{resoft} at the command line.
      }
    \end{enumerate}
  \end{itemize}
  {\tiny
  \begin{alertblock}{}
    \begin{verbatim}
[apacheco@eric2 ~]$ cat .soft
#   
# This is the .soft file.
...
+mvapich-1.1-intel-11.1
+intel-fc-11.1
+intel-cc-11.1
+espresso-4.3.1-intel-11.1-mvapich-1.1
+gaussian-09
+lmto-intel-11.1
+nciplot-intel-11.1
+gaussview-4.1.2
+jmol-12
+vmd-1.8.6
+xcrysden-1.5.24-gcc-4.3.2
+tcl-8.5.8-intel-11.1
+gamess-12Jan2009R1-intel-11.1
+nwchem-5.1.1-intel-11.1-mvapich-1.1
+cpmd-3.11.1-intel-11.1-mvapich-1.1
@default
[apacheco@eric2 ~]$ resoft
    \end{verbatim}
  \end{alertblock}
  }
\end{frame}

\begin{frame}[fragile]
  \frametitle{\small Querying a Softenv Key}
  \begin{itemize}
    \item \texttt{soft-dbq} shows which variables are set by a SOFTENV key
  \end{itemize}
  {\tiny
  \vspace{-0.2cm}
  \begin{alertblock}{}
    \begin{verbatim}
[apacheco@eric2 ~]$ soft-dbq +amber-11-intel-11.1-mvapich-1.1

This is all the information associated with
the key or macro +amber-11-intel-11.1-mvapich-1.1.

-------------------------------------------

Name: +amber-11-intel-11.1-mvapich-1.1
Description: @types: Applications @name: Amber @build: amber-11-intel-11.1-mvapich-1.1
 ...
Exists on: Linux 

-------------------------------------------

On the Linux architecture,
the following will be done to the environment:

  The following environment changes will be made:
    AMBERHOME = /usr/local/packages/amber/11/intel-11.1-mvapich-1.1
    LD_LIBRARY_PATH = ${LD_LIBRARY_PATH}:/usr/local/compilers/Intel/mkl-10.2/lib/em64t
    PATH = ${PATH}:/usr/local/packages/amber/11/intel-11.1-mvapich-1.1/exe

-------------------------------------------
  \end{verbatim}
  \end{alertblock}
  }
\end{frame}

\begin{frame}<1,2>
  \frametitle{\small Exercise 2: Using Softenv}
  \begin{eblock}{ }
    \begin{itemize}
      \only<1->{\item Find the key for VISIT (a visualization package).}
      \visible<2>{\item[] \texttt{softenv -k visit}}
      \only<1->{\item Check what variables are set through the key.}
      \visible<2>{\item[] \texttt{soft-dbq +visit}}
      \only<1->{\item Set up your environment to use VISIT.}
      \visible<2>{\item[] \texttt{soft add +visit}}
      \only<1->{\item Check if the variables are correctly set by using \texttt{which visit}.}
      \visible<2>{\item[] \texttt{/usr/local/packages/visit/bin/visit}}
    \end{itemize}
  \end{eblock}
\end{frame}


\begin{frame}
  \frametitle{\small Compilers}
  \begin{bblock}{}
    \begin{center}
      \begin{tabular}{|c|c|c|c|c|}
	\hline
	\multirow{2}{*}{Language} & \multicolumn{3}{c|}{Linux Cluster} & AIX Clusters\\
	\cline{2-5}
	& Intel & PGI & GNU & XL \\
	\hline
	Fortran & ifort & pgf77,pgf90 & gfortran & xlf,xlf90\\\hline
	C & icc & pgcc & gcc & xlc\\\hline
	C++ & icpc & pgCC & g++ & xlC\\
	\hline
      \end{tabular}
    \end{center}
  \end{bblock}
  \begin{itemize}
    \item Usage: <compiler> <options> <your\_code>
    \begin{enumerate}
      \item[$\vardiamond$] Example: icc -O3 -o myexec mycode.c
    \end{enumerate}
    \item Some compilers options are architecture specific
    \begin{enumerate}
      \item[$\vardiamond$] Linux: EM64T, AMD64 or X86\_64
      \item[$\vardiamond$] AIX: power5,power7 or powerpc
    \end{enumerate}
  \end{itemize}
\end{frame}

\begin{frame}
  \frametitle{\small Compilers for MPI programs}
  \begin{colorblock}{white}{blue!30!black}{black}{blue!15!white}{}
    \begin{center}
      \begin{tabular}{|c|c|c|}
	\hline
	Language & {Linux Cluster} & AIX Clusters\\
	\hline
	Fortran & mpif77,mpif90 & mpxlf,mpxlf90\\\hline
	C & mpicc & mpcc\\\hline
	C++ & mpiCC & mpCC\\
	\hline
      \end{tabular}
    \end{center}
  \end{colorblock}
  \begin{itemize}
    \item Usage: <compiler> <options> <your\_code>
    \begin{enumerate}
      \item[$\vardiamond$] Example: mpif90 -O2 -o myexec mycode.f90
    \end{enumerate}
    \item On Linux clusters
    \begin{enumerate}
      \item[$\vardiamond$] Only one compiler for each language
      \item[$\vardiamond$] There is no intel\_mpicc or pg\_mpicc
    \end{enumerate}
    \item There are many different versions of MPI compilers on Linux clusters
    \begin{enumerate}
      \item[$\vardiamond$]  Each of them is built around a specific compiler
      \item[$\vardiamond$]  Intel, PGI or GNU
    \end{enumerate}
  \end{itemize}
\end{frame}

\begin{frame}[fragile]
  \frametitle{\small Compiling and Running MPI programs}
  \begin{itemize}
    \item It is extremely important to compile and run you code with the same version!!!
    \item Use the default version if possible
    \item These MPI compilers are actually wrappers
    \begin{enumerate}
      \item[$\vardiamond$]  They still use the compilers we've seen on the previous slide
      \begin{enumerate}
	\item[$\bigstar$]  Intel, PGI or GNU
      \end{enumerate}
      \item[$\vardiamond$]  They take care of everything we need to build MPI codes
      \begin{enumerate}
	\item[$\bigstar$]  Head files, libraries etc.
      \end{enumerate}
      \item[$\vardiamond$]  What they actually do can be reveal by the \texttt{-show} option
    \end{enumerate}
  \end{itemize}
  {\tiny
  \begin{alertblock}{}
    {%\color{black}
    \begin{verbatim}
[apacheco@eric2 ~]$ mpif90 -show
ln -s /usr/local/packages/mvapich/1.1/intel-11.1/include/mpif.h mpif.h
ifort -fPIC -L/usr/local/ofed/lib64 -Wl,-rpath-link -Wl, \
   /usr/local/packages/mvapich/1.1/intel-11.1/lib/shared \
  -L/usr/local/packages/mvapich/1.1/intel-11.1/lib/shared \
  -L/usr/local/packages/mvapich/1.1/intel-11.1/lib \
  -lmpichf90nc -lmpichfarg -lmpich -L/usr/local/ofed/lib64 \
  -Wl,-rpath=/usr/local/ofed/lib64 -libverbs -libumad -lpthread -lpthread -lrt -limf
rm -f mpif.h
    \end{verbatim}
    }
  \end{alertblock}
  }
\end{frame}

\begin{frame}
  \frametitle{\small Application Packages}
  \begin{itemize}
    \item Installed under \texttt{/usr/local/packages}
    \item Most of them managed by SOFTENV
    \begin{itemize}
      \item[$\vardiamond$] Numerical and utility libraries
      \begin{enumerate}
	\item[$\blacksquare$] FFTW, HDF5, NetCDF, PetSc, Intel MKL
      \end{enumerate}
      \item[$\vardiamond$] Computational Chemistry
      \begin{enumerate}
	\item[$\blacksquare$] Amber, CPMD, Gaussian, GAMESS, Gromacs, LAMMPS, NAMD, NWCHEM
      \end{enumerate}
      \item[$\vardiamond$] Visualization
      \begin{enumerate}
	\item[$\blacksquare$] GaussView, VisIt, VMD 
      \end{enumerate}
      \item[$\vardiamond$] Profiling/debugging tools
      \begin{enumerate}
	\item[$\blacksquare$] DDT, Tau, TotalView
      \end{enumerate}
      \item[$\vardiamond$] MPI Implementation
      \begin{enumerate}
	\item[$\blacksquare$] mvapich, mvapich2, mpich, openmpi
      \end{enumerate}
      \item[$\vardiamond$] $\cdots$
    \end{itemize}
  \end{itemize}
\end{frame}

\begin{frame}
  \frametitle{\small Exercise 3: Compiling a code}
  \begin{eblock}{}
    \begin{enumerate}
      \item Serial Code
      \begin{itemize}
	\only<1->{\item On Linux cluster, add the soft keys for either Intel (\texttt{+intel-fc-11.1}) or GCC (\texttt{+gcc-4.3.2})}
	\only<1->{\item Compile \texttt{hello.f90} with a compiler of your choice}
	\visible<2>{\item[] \texttt{ifort -o hello hello.f90}}
	\visible<2>{\item[] \texttt{xlf90 -o hello hello.f90}}
	\only<1->{\item Run the executable from the command line}
	\visible<2>{\item[] \texttt{./hello}}
      \end{itemize}
      \item Parallel Code
      \begin{itemize}
	\only<1->{\item On Linux cluster, find the appropriate key for mpi implementation of the above compiler}
	\only<1->{\item Compile \texttt{hello\_mpi.f90}}
	\visible<2>{\item[] \texttt{mpif90 -o hellompi hello\_mpi.f90}}
	\visible<2>{\item[] \texttt{mpxlf90 -o hellompi hello\_mpi.f90}}
	\only<1->{\item Do Not run the parallel code, we'll use a script to submit to a job manager}
      \end{itemize}
    \end{enumerate}
  \end{eblock}
\end{frame}

\section{Job Management}
\subsection{Queues}
\begin{frame}
  \frametitle{\small The Cluster Environment}
  \vspace{-0.5cm}
  \begin{columns}
    \column{4cm}
    \begin{itemize}
      \item A cluster is a group of computers (nodes) that works together closely
      \item Type of nodes
      \begin{enumerate}
	{\scriptsize
	\item[$\vardiamond$] Head node
	\item[$\vardiamond$] Multiple Compute nodes
	}
      \end{enumerate}
      \item Multi User Environment
      \item Each user may have multiple jobs running simultaneously.
    \end{itemize}
    \column{7cm}
    \begin{center}
      \includegraphics[width=1.05\textwidth,clip=true]{cluster}
    \end{center}
  \end{columns}
\end{frame}

\begin{frame}
  \frametitle{\small Batch Queuing System}
  \begin{bblock}{}
    \begin{itemize}
      \item A software that manages resources (CPU time, memory, etc) and schedules job execution
      \begin{enumerate}
	\item[$\vardiamond$] Linux Clusters: Portable Batch System (PBS)
	\item[$\vardiamond$] AIX Clusters: Loadleveler
      \end{enumerate}
      \item A job can be considered as a user's request to use a certain amount of resources for a certain amount of time
      \item The batch queuing system determines
      \begin{enumerate}
	\item The order jobs are executed
	\item On which node(s) jobs are executed
      \end{enumerate}
    \end{itemize}
  \end{bblock}
\end{frame}

\begin{frame}
  \frametitle{\small A Simplified View of Job Scheduling}
  \begin{columns}
    \column{7cm}
    \vspace{-1cm}
    \begin{center}
      \includegraphics[width=0.85\textwidth,clip=true]{JobSchedule-1}
    \end{center}
    \column{5cm}
    \vspace{-0.5cm}
    \begin{bblock}{}
      \begin{itemize}
	\item Map jobs onto the node-time space
	\begin{itemize}
	  \item Assuming CPU time is the only resource
	\end{itemize}
	\item Need to find a balance between
	\begin{itemize}
	  \item Honoring the order in which jobs are received
	  \item Maximizing resource utilization
	\end{itemize}
      \end{itemize}
    \end{bblock}
  \end{columns}
\end{frame}

\begin{frame}
  \frametitle{\small Backfilling}
  \begin{columns}
    \column{7cm}
    \vspace{-1cm}
    \begin{center}
      \includegraphics[width=0.85\textwidth,clip=true]{JobSchedule-2}
    \end{center}
    \column{5cm}
    \vspace{-0.5cm}
    \begin{bblock}{}
      \begin{itemize}
	\item A strategy to improve utilization
	\begin{itemize}
	  \item Allow a job to jump ahead of others when there are enough idle nodes
	  \item Must not affect the estimated start time of the job with the highest priority
	\end{itemize}
	\item Enabled on all LONI clusters
      \end{itemize}
    \end{bblock}
  \end{columns}
\end{frame}

\begin{frame}
  \frametitle{\small How much time Should I request?}
  \begin{columns}
    \column{0.5\textwidth}
    \vspace{-1cm}
    \begin{center}
      \includegraphics[width=0.9\textwidth,clip=true]{JobSchedule-3}
    \end{center}
    \column{0.5\textwidth}
    \vspace{-1cm}
    \begin{center}
      \includegraphics[width=0.9\textwidth,clip=true]{JobSchedule-4}
    \end{center}
  \end{columns}
  \begin{bblock}{}
    \begin{itemize}
      \item Ask for an amount of time that is
      \begin{itemize}
	\item Long enough for your job to complete
	\item As short as possible to increase the chance of backfilling
      \end{itemize}
    \end{itemize}
  \end{bblock}
\end{frame}

\begin{frame}
  \frametitle{\small Job Queues}
  \begin{bblock}{}
    \begin{itemize}
      \item There are more than one job queue
      \item Each job queue differs in
      \begin{itemize}
	\item Number of available nodes
	\item Maximum run time
	\item Maximum running jobs per user
      \end{itemize}
      \item The main purpose is to maximize utilization
    \end{itemize}
  \end{bblock}
\end{frame}

\begin{frame}
  \frametitle{\small Queue Characteristics: LONI Linux Clusters}
  \begin{columns}
    \column{12cm}
    \vspace{-0.5cm}
    \begin{bblock}{QueenBee}
      {\scriptsize
      \begin{center}
	\begin{tabular}{|m{0.1\textwidth}|m{0.1\textwidth}|m{0.1\textwidth}|m{0.12\textwidth}|m{0.1\textwidth}|m{0.25\textwidth}|}
	  \hline
	  Queue & Max Runtime & Total number of nodes & Max running jobs per user & Max nodes per job & Use \\
	  \hline
	  workq & \multirow{4}{*}{2 days} & 530 & \multirow{2}{*}{8} & 128 & Unpreemptable \\
	  \cline{1-1}\cline{3-3}\cline{5-6}
	  checkpt & & 668 & & 256 & preemptable\\
	  \cline{1-1}\cline{3-6}
	  preempt & & 668 & \multicolumn{2}{c|}{NA} & Requires permission \\
	  \cline{1-1}\cline{3-6}
	  priority & & 668 & \multicolumn{2}{c|}{NA} & Requires permission \\\hline 
	\end{tabular}
      \end{center}
      }
    \end{bblock}
    \begin{bblock}{Other Clusters}
      {\scriptsize
      \begin{center}
	\begin{tabular}{|m{0.1\textwidth}|m{0.1\textwidth}|m{0.1\textwidth}|m{0.12\textwidth}|m{0.1\textwidth}|m{0.25\textwidth}|}
	  \hline
	  Queue & Max Runtime & Total number of nodes & Max running jobs per user & Max nodes per job & Use \\
	  \hline
	  single & 14 days & 16 & 64 & 1 & Single processor jobs \\\hline
	  workq & \multirow{4}{*}{3 days} & 96 & \multirow{2}{*}{8} & 40 & Unpreemptable \\
	  \cline{1-1}\cline{3-3}\cline{5-6}
	  checkpt & & 128 & & 64 & preemptable\\
	  \cline{1-1}\cline{3-6}
	  preempt & & 64 & \multicolumn{2}{c|}{NA} & Requires permission \\
	  \cline{1-1}\cline{3-6}
	  priority & & 64 & \multicolumn{2}{c|}{NA} & Requires permission\\
	  \hline
	\end{tabular}
      \end{center}
      }
    \end{bblock}
  \end{columns}
\end{frame}

\subsection{Job Manager Commands}
\begin{frame}
  \frametitle{\small Basic Job Manager Commands}
  \begin{bblock}{}
    \begin{itemize}
      \item Queue querying
      \begin{itemize}
	\item Check how busy the cluster is
      \end{itemize}
      \item Job submission
      \item Job monitoring
      \begin{itemize}
	\item Check job status (estimated start time, remaining run time, etc)
      \end{itemize}
      \item Job manipulation
      \begin{itemize}
	\item Cancel/Hold jobs
      \end{itemize}
    \end{itemize}
  \end{bblock}
\end{frame}

\begin{frame}[fragile]
  \frametitle{\small Queue Querying: Linux Clusters}
  \begin{itemize}
    \item \texttt{qfree}: show number of free,busy and queued nodes
    \item \texttt{qfreeloni}: run \texttt{qfree} on all LONI Linux clusters
  \end{itemize}
  \begin{eblock}{}
    {\tiny
    \begin{verbatim}
[apacheco@eric2 ~]$ qfree
PBS total nodes: 128,  free: 49,  busy: 79,  down: 0,  use: 61%
PBS workq nodes: 96,  free: 40,  busy: 28,  queued: 0
PBS checkpt nodes: 104,  free: 40,  busy: 35,  queued: 0
PBS single nodes: 32,  free: 9 *36,  busy: 16,  queued: 366
[apacheco@eric2 ~]$ qfreeloni
-------- qb --------
PBS total nodes: 668,  free: 3,  busy: 647,  down: 18,  use: 96%
PBS workq nodes: 530,  free: 0,  busy: 278,  queued: 367
PBS checkpt nodes: 668,  free: 1,  busy: 369,  queued: 770
-------- eric --------
PBS total nodes: 128,  free: 49,  busy: 79,  down: 0,  use: 61%
PBS workq nodes: 96,  free: 40,  busy: 28,  queued: 0
PBS checkpt nodes: 104,  free: 40,  busy: 35,  queued: 0
PBS single nodes: 32,  free: 9 *36,  busy: 16,  queued: 366
-------- louie --------
PBS total nodes: 128,  free: 44,  busy: 83 *2,  down: 1,  use: 64%
PBS workq nodes: 104,  free: 40,  busy: 0,  queued: 0
PBS checkpt nodes: 128,  free: 44,  busy: 82,  queued: 50
PBS single nodes: 32,  free: 7 *26,  busy: 2,  queued: 0
-------- oliver --------
PBS total nodes: 128,  free: 74,  busy: 52,  down: 2,  use: 40%
PBS workq nodes: 62,  free: 8,  busy: 11,  queued: 0
...
    \end{verbatim}
    }
  \end{eblock}
\end{frame}

\subsection{Job Types}
\begin{frame}[overlayarea]
  \frametitle{\small Job Types}
  \only<1>{
  \begin{bblock}{Interactive Jobs}
    \begin{itemize}
      \item Set up an interactive environment on compute nodes for users
      \begin{itemize}
	\item[$\blacksquare$] Advantage: can run programs interactively
	\item[$\blacksquare$] Disadvantage: must be present when job starts
      \end{itemize}
      \item Purpose:  testing and debugging code. \textbf{Do not run jobs on head node!!!}
      \item[] \texttt{qsub -I -V -l walltime=<hh:mm:ss>,nodes=<\# of nodes>:ppn=cpu -A <your allocation> -q <queue name>}
      \item On QueenBee, cpu=8
      \item Other LONI Clusters:  cpu=4 (parallel jobs) or cpu=1 (single queue)
      \item To enable X-forwarding: add \texttt{-X}
    \end{itemize}
  \end{bblock}
  }
  \only<2>{
  \begin{bblock}{Batch Jobs}
    \begin{itemize}
      \item Executed using a batch script without user intervention
      \begin{itemize}
	\item[$\blacksquare$] Advantage: system takes care of running the job
	\item[$\blacksquare$] Disadvantage: can change sequence of commands after submission
      \end{itemize}
      \item Useful for Production runs
      \item[] \texttt{qsub <job script>}
      \item[] \texttt{llsubmit <job script>}
    \end{itemize}
  \end{bblock}
  }
\end{frame}

\subsection{Job Submission Scripts}
\begin{frame}[fragile]
  \frametitle{\small PBS Job Script: Parallel Jobs}
  \begin{columns}
  \column{7.2cm}
  \begin{eblock}{}
    {\footnotesize
    \begin{verbatim}
#!/bin/bash
#PBS -l nodes=4:ppn=4
#PBS -l walltime=24:00:00
#PBS -N myjob
#PBS -o <file name>
#PBS -e <file name>
#PBS -q checkpt	
#PBS -A <loni_allocation>
#PBS -m e
#PBS -M <email address>

<shell commands>			
mpirun  -machinefile $PBS_NODEFILE \
 -np 16 <path_to_executable> <options>
<shell commands>
    \end{verbatim}
  }
  \end{eblock}
  \column{4.5cm}
  \begin{ablock}{}
    {\footnotesize %\color{red}
    \begin{verbatim}
Shell being used
# of nodes & processors
Maximum walltime
Job name
standard output
standard error
Queue name
Allocation name
Send mail when job ends 
to this address

shell commands
run parallel job

shell commands
      \end{verbatim}
      }
    \end{ablock}
  \end{columns}
\end{frame}

\begin{frame}[fragile]
  \frametitle{\small PBS Job Script: Serial Jobs}
  \begin{columns}
    \column{7.2cm}
    \begin{eblock}{}
      {\footnotesize
      \begin{verbatim}
#!/bin/bash
#PBS -l nodes=1:ppn=1
#PBS -l walltime=24:00:00
#PBS -N myjob
#PBS -o <file name>
#PBS -e <file name>
#PBS -q single	
#PBS -A <loni_allocation>
#PBS -m e
#PBS -M <email address>

<shell commands>			
<path_to_executable> <options>
<shell commands>
      \end{verbatim}
    }
    \end{eblock}
    \column{4.5cm}
    \begin{ablock}{}
      {\footnotesize %\color{red}
      \begin{verbatim}
Shell being used
# of nodes & processors
Maximum walltime
Job name
standard output
standard error
Use single queue 
Allocation name
Send mail when job ends 
to this address

shell commands
run parallel job
shell commands
      \end{verbatim}
      }
    \end{ablock}
  \end{columns}
\end{frame}

\begin{frame}
  \frametitle{\small Exercise 4: Job Submission}
  \begin{eblock}{}
    \begin{itemize}
      \item Write a job submission script to execute the \texttt{hellompi} program.
      \item Submit the script to the job manager.
    \end{itemize}
  \end{eblock}
\end{frame}


\subsection{Job Monitoring \& Manipulation}
\begin{frame}
  \frametitle{\small Job Monitoring}
  \begin{bblock}{Linux Clusters}
    \begin{itemize}
      \item \texttt{showstart <job id>}
      \begin{enumerate}
	\item[$\vardiamond$] Check estimated time when job can start
      \end{enumerate}
      \item When can the estimated time change
      \begin{enumerate}
	\item[$\vardiamond$] Higher priority job gets submitted
	\item[$\vardiamond$] Running jobs terminate earlier than time requested
	\item[$\vardiamond$] System has trouble starting your job
      \end{enumerate}
      \item \texttt{qstat <options> <job id>}
      \begin{enumerate}
	\item[$\vardiamond$] Show information on job status
	\item[$\vardiamond$] All jobs displayed if \texttt{<job id>} is omitted
	\item[$\vardiamond$] \texttt{qstat -u <username>}: Show jobs belonging to \texttt{<username>}
	\item[$\vardiamond$] \texttt{qstat -a <job id>}: Displat in an alternative format
      \end{enumerate}
      \item \texttt{qshow <job id>}
      \begin{enumerate}
	\item[$\vardiamond$] Show information of running job \texttt{<job id>}: node running on and CPU load
      \end{enumerate}
    \end{itemize}
  \end{bblock}
\end{frame}

\begin{frame}
  \frametitle{\small Job Manipulation}
  \begin{bblock}{Linux Clusters}
    \begin{itemize}
      \item \texttt{qdel <job id>}
      \begin{enumerate}
	\item[$\vardiamond$] Cancel a running or queued job
      \end{enumerate}
      \item \texttt{qhold <job id>}
      \begin{enumerate}
	\item[$\vardiamond$] Put a queued job on hold
      \end{enumerate}
      \item \texttt{qrls <job id>}
      \begin{enumerate}
	\item[$\vardiamond$] Resume a held job
      \end{enumerate}
    \end{itemize}
  \end{bblock}

  \begin{bblock}{AIX Clusters}
    \begin{itemize}
      \item \texttt{llcancel <job id>}
      \begin{enumerate}
	\item[$\vardiamond$] Cancel a running or queued job
      \end{enumerate}
      \item \texttt{llhold <job id>}
      \begin{enumerate}
	\item[$\vardiamond$] Put a queued job on hold
      \end{enumerate}
      \item \texttt{llhold -r <job id>}
      \begin{enumerate}
	\item[$\vardiamond$] Resume a held job
      \end{enumerate}
    \end{itemize}
  \end{bblock}
\end{frame}

\section{HPC Help}
\begin{frame}
  \frametitle{\small Additional Help}
  \begin{itemize}
    \item User's Guide
    \begin{enumerate}
      \item[$\vardiamond$]HPC: \url{http://www.hpc.lsu.edu/help}
      \item[$\vardiamond$]LONI: \url{https://docs.loni.org/wiki/Main_Page}
    \end{enumerate}
    \item Contact us
    \begin{enumerate}
      \item[$\vardiamond$]Email ticket system: sys-help@loni.org
      \item[$\vardiamond$]Telephone Help Desk: 225-578-0900
      \item[$\vardiamond$]Walk-in consulting session at Middleton Library
      \begin{enumerate}
	\item[$\bigstar$]Tuesdays and Thursdays only
      \end{enumerate}
      \item[$\vardiamond$]Instant Messenger (AIM, Yahoo Messenger, Google Talk)
      \begin{enumerate}
	\item[$\bigstar$]Add "lsuhpchelp"
      \end{enumerate}
    \end{enumerate}
  \end{itemize}
\end{frame} 

\begin{frame}
  \frametitle{\small  }
  \vspace{1.2cm}
  \begin{eblock}{}
    \centering{
      \Huge{
	\vspace{0.5cm}
	THE END\\
	\vspace{1cm}
	Questions, Comments ???
      }
    }
  \end{eblock}
\end{frame}

\end{document}

